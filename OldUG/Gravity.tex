\chapter{Gravity Unit}
\label{chp:Gravity}


\section{Introduction}

The \unit{Gravity} unit supplied with \flashx computes gravitational source
terms for the code.
These source terms can take the form of the gravitational potential
$\phi({\bf x})$ or the gravitational acceleration ${\bf g}({\bf x})$,
\begin{equation}
{\bf g}({\bf x}) = -\nabla \phi({\bf x})\ .
\end{equation}
The gravitational field can be externally imposed or self-consistently
computed from the gas density via the Poisson equation,
\begin{equation}
\label{Eqn:Poisson}
\nabla^2\phi({\bf x}) = 4\pi G \rho({\bf x})\ ,
\end{equation}
where $G$ is Newton's gravitational constant.
In the latter case, either periodic or isolated boundary conditions can
be applied.

%--------------------------------------------------------------------------

\section{Externally Applied Fields}
\label{Sec:GravityExternal}
The Flash-X distribution includes three externally applied gravitational
fields, along with a placeholder module for you to create your own. Each
provides the acceleration vector ${\bf g}({\bf x})$ directly, without using
the gravitational potential $\phi({\bf x})$ (with the exception of
\code{UserDefined}, see below).

When building an application that uses an external, time-independent
\unit{Gravity} implementation, no additional storage in \code{unk}
for holding gravitational potential or accelerations is needed
or defined.

\subsection{Constant Gravitational Field}
This implementation creates
a spatially and temporally constant field parallel to one of the
coordinate axes. The magnitude and direction of the field can be set
at runtime.  This unit is called \code{Gravity/\-GravityMain/\-Constant}.

\subsection{Plane-parallel Gravitational field}
This \code{PlanePar} version
implements a time-constant  gravitational field that is parallel to
one of the coordinate axes and falls off with the square of the
distance from a fixed location. The field is assumed to be generated
by a point mass or by a spherically symmetric mass distribution. A
finite softening length may optionally be applied.

This type of gravitational field
is useful when the computational domain is large enough in the
direction radial to the field source that the field is not
approximately constant, but the domain's dimension perpendicular to
the radial direction is small compared to the distance to the source.
In this case the angular variation of the field direction may be
ignored. The \code{PlanePar} field is cheaper to compute than the
\code{PointMass} field described below,
since no fractional powers of the distance are
required. The acceleration vector is parallel to one of the coordinate
axes, and its magnitude drops off with distance along that axis as the
inverse distance squared. Its magnitude and direction are independent
of the other two coordinates.

\subsection{Gravitational Field of a Point Mass}
This \code{PointMass} implementation
describes the gravitational field due to a point mass at a fixed
location. A finite softening length may optionally be applied. The
acceleration falls off with the square of the distance from a given
point. The acceleration vector is everywhere directed toward this point.

\subsection{User-Defined Gravitational Field}
The \code{UserDefined} implementation is a placeholder module for the user to
create their own external gravitational field. All of the subroutines in this
module are stubs, and the user may copy these stubs to their setup directory to
write their own implementation, either by specifying the gravitational
acceleration directly or by specifying the gravitational potential and taking
its gradient. If your user-defined gravitational field is time-varying, you
may also want to set \code{PPDEFINE Flash-X\_GRAVITY\_TIMEDEP} in your setup's
Config file.

\section{Self-gravity}
\label {Sec:GravitySelfgravity}
The self-consistent gravity algorithm supplied with Flash-X computes the
Newtonian gravitational field produced by the matter. The produced
potential function satisfies Poisson's equation
\eqref{Eqn:Poisson}. This unit's implementation
can also return the acceleration field ${\bf g}({\bf x})$
computed by finite-differencing the potential using the expressions
\begin{eqnarray}
\nonumber
g_{x;ijk} &= {1\over2\Delta x}\left(\phi_{i-1,j,k} - \phi_{i+1,j,k}\right) +
  {\cal O}(\Delta x^2) \\
g_{y;ijk} &= {1\over2\Delta y}\left(\phi_{i,j-1,k} - \phi_{i,j+1,k}\right) +
  {\cal O}(\Delta y^2) \\
\nonumber
g_{z;ijk} &= {1\over2\Delta z}\left(\phi_{i,j,k-1} - \phi_{i,j,k+1}\right) +
  {\cal O}(\Delta z^2) \ .
\end{eqnarray}
In order to preserve the second-order accuracy of these expressions at
jumps in grid refinement, it is important to use quadratic interpolants
when filling guard cells at such locations. Otherwise, the truncation error
of the interpolants will produce unphysical forces at these block boundaries.

Two algorithms are available for solving the Poisson equations:
\code{Gravity/\-GravityMain/\-Multipole} and \code{Gravity/\-GravityMain/\-Multigrid}.
The initialization routines for these algorithms are contained in the \code{Gravity}
unit, but the actual implementations are contained below the \code{Grid} unit due
to code architecture constraints.

The multipole-based solver described in \secref{Sec:GridSolversMultipole}
for self gravity is appropriate for spherical or nearly-spherical mass
distributions with isolated boundary conditions. For non-spherical
mass distributions higher order moments of the solver must be used. Note that
storage and CPU costs scale roughly as the square of number of moments
used, so it is best to use this solver only for nearly spherical
matter distributions.

The multigrid solver described in \secref{Sec:GridSolversMultigrid}
is appropriate for general mass distributions and can solve problems with
more general boundary conditions.

The tree solver described in \ref{Sec:GridSolversBHTree} is appropriate for
general mass distributions and can solve problems with both isolated and
periodic boundary conditions set independently in individual directions.


\subsection{Coupling Gravity with Hydrodynamics}

The gravitational field couples to the Euler equations only through the
momentum and energy equations. If we define the total energy density as
\begin{equation}
\rho E \equiv {1\over 2}\rho v^2 + \rho\epsilon\ ,
\end{equation}
where $\epsilon$ is the specific internal energy, then the gravitational
source terms for the momentum and energy equations are $\rho{\bf g}$ and
$\rho{\bf v}\cdot{\bf g}$, respectively. Because of the variety of ways
in which different hydrodynamics schemes treat these source terms,
the gravity module only supplies the potential $\phi$ and acceleration ${\bf g}$,
leaving the implementation
of the fluid coupling to the hydrodynamics module.
Finite-difference and finite-volume hydrodynamic schemes apply the source terms
in their advection steps,
sometimes at multiple intermediate timesteps and sometimes using
staggered meshes for vector quantities like ${\bf v}$ and ${\bf g}$.

For example, the PPM algorithm supplied with Flash-X uses the following
update steps to obtain the momentum and energy in cell $i$ at timestep
$n+1$
\begin{eqnarray}
\nonumber
(\rho v)_i^{n+1} & =  (\rho v)_i^n + {\Delta t\over 2} g_i^{n+1}
  \left(\rho_i^n + \rho_i^{n+1}\right)\\
(\rho E)_i^{n+1} & =  (\rho E)_i^n + {\Delta t\over 4} g_i^{n+1}
  \left(\rho_i^n + \rho_i^{n+1}\right)\left(v_i^n + v_i^{n+1}\right)\ .
\end{eqnarray}
Here $g_i^{n+1}$ is obtained by extrapolation from $\phi_i^{n-1}$ and
$\phi_i^n$.
The $\code{Poisson}$ gravity implementation supplies a mesh variable to contain the
potential from the previous timestep; future releases of Flash-X may
permit the storage of several time levels of this quantity for
hydrodynamics algorithms that require more steps.
Currently, ${\bf g}$ is computed at cell centers.

Note that finite-volume schemes do
not retain explicit conservation of momentum and energy when gravity
source terms are added. Godunov schemes such as PPM, require an
additional step in order to preserve second-order time accuracy. The
gravitational acceleration component $g_i$ is fitted by interpolants along
with the other state variables, and these interpolants are used to
construct characteristic-averaged values of ${\bf g}$ in each cell. The
velocity states $v_{L,i+1/2}$ and $v_{R,i+1/2}$, which are used as
inputs to the Riemann problem solver, are then corrected to account
for the acceleration using the following expressions
\begin{eqnarray}
\nonumber
v_{L,i+1/2} &\rightarrow& v_{L,i+1/2} + {\Delta t\over
4}\left(g^+_{L,i+1/2} +
  g^-_{L,i+1/2}\right)\\
v_{R,i+1/2} &\rightarrow& v_{R,i+1/2} + {\Delta t\over
4}\left(g^+_{R,i+1/2} +
  g^-_{R,i+1/2}\right)~.
\end{eqnarray}
Here $g^\pm_{X,i+1/2}$ is the acceleration averaged using the interpolant
on
the $X$ side of the interface ($X=L,R$) for $v\pm c$ characteristics, which
bring material to the interface between cells $i$ and $i+1$ during the
timestep.

\subsection{Tree Gravity}

The \code{Tree} implementation of the gravity unit in \code{physics/Gravity/GravityMain/Poisson/BHTree} is
meant to be used together with the tree solver implementation
\texttt{Grid/GridSolvers/BHTree/Wunsch}. It either calculates the gravitational
potential field which is subsequently differentiated in subroutine
\api{physics/Gravity/Gravity\_accelOneRow} to obtain the gravitational acceleration, or it
calculates the gravitational acceleration directly. The latter approach is more
accurate, because the error due to numerical differentiation is avoided,
however, it consumes more memory for storing three components of the
gravitational acceleration. The direct acceleration calculation can be switched
on by specifying \code{bhtreeAcc=1} as a command line argument of the setup
script.

The gravity unit provides subroutines for building and walking the tree called
by the tree solver. In this version, only monopole moments (node masses) are
used for the potential/acceleration calculation. It also defines new multipole
acceptance criteria (MACs) that estimate the error in gravitational acceleration
of a contribution of a single node to the potential (hereafter partial error)
much better than purely geometrical MAC defined in the tree solver. They are:
(1) the approximate partial error (APE), and (2) the maximum partial error
(MPE). The first one is based on an assumption that the partial error is
proportional to the multipole moment of the node. The node is accepted for
calculation of the potential if
\begin{equation}
D^{m+2} > \frac{GMS_\mathrm{node}^m}{\Delta a_\mathrm{p,APE}}
\end{equation}
where $D$ is distance between the {\em point-of-calculation} and the node, $M$
is the node mass, $S_\mathrm{node}$ is the node size, $m$ is a degree of the
multipole approximation and $\Delta a_\mathrm{p,APE}$ is the requested maximum error in
acceleration (controlled by runtime parameter \rpi{Gravity/grv\_bhAccErr}). Since only
monopole moments are used for the potential calculation, the most reasonable
choice of $m$ seems to be $m=2$. This MAC is similar to the one used in Gadget2
(see Springel, 2005, MNRAS, 364, 1105).

The second MAC (maximum partial error, MPE) calculates the error in acceleration
of a single node contribution $\Delta a_\mathrm{p,MPE}$ according to formula 9
from Salmon\&Warren (1994; see this paper for details):
\begin{equation}
\Delta a_\mathrm{p,MPE} \le \frac{1}{D^2}\frac{1}{(1-S_\mathrm{node}/D)^2}\left(
\frac{3\lceil B_2\rceil}{D^2} - \frac{2\lfloor B_3 \rfloor}{D^3}\right)
\end{equation}
where $B_n = \Sigma_i m_i |\mathbf{r}_i-\mathbf{r}_0|^n $ where $m_i$ and
$\mathbf{r}_i$ are masses and positions of individual grid cells within the node
and $\mathbf{r}_0$ is the node mass center position. Moment $B_2$ can be easily
determined during the tree build, moment $B_3$ can be estimated as $B_3^2 \ge
B_2^3/M$. The maximum allowed partial error in gravitational acceleration is
controlled by runtime parameters \rpi{Gravity/grv\_bhAccErr} and
\rpi{Gravity/grv\_bhUseRelAccErr} (see \ref{Sec:GravityBHTreeUsing}).

During the tree walk, subroutine \api{physics/Gravity/Gravity\_bhNodeContrib} adds contributions
of tree nodes to the gravitational potential or acceleration fields. In case of
the potential, the contribution is 
\begin{equation}
\Phi = -\frac{GM}{|\vec{r}|}
\end{equation}
if \texttt{isolated} boundary conditions are used, or
\begin{equation}
\Phi = -GMf_\mathrm{EF,\Phi}(\vec{r})
\end{equation}
if periodic \texttt{periodic} or \texttt{mixed} boundary conditions are used. In
case of the acceleration, the contributions are 
\begin{equation}
\vec{a}_g = \frac{GM\vec{r}}{|\vec{r}|^3}
\end{equation}
for \texttt{isolated} boundary conditions, or
\begin{equation}
\vec{a}_g = GMf_\mathrm{EF,a}(\vec{r})
\end{equation}
for \texttt{periodic} or \texttt{mixed} boundary conditions. In In the above
formulae, $G$ is the constant of gravity, $M$ is the node mass, $\vec{r}$ is the
position vector between {\em point-of-calculation} and the node mass center and
$f_\mathrm{EF,\Phi}$ and $f_\mathrm{EF,a}$ are the Ewald fields for the
potential and the acceleration (see below).

Boundary conditions can be isolated or periodic, set independently for each
direction. If they are periodic at least in one direction, the Ewald method is
used for the potential calculation (Ewald, P.~P., 1921, Ann.\ Phys.\ 64, 253).
The original Ewald method is an efficient method for computing gravitational
field for problems with periodic boundary conditions in three directions. Ewald
speeded up evaluation of the gravitational potential by splitting it into two parts,
$Gm/r=Gm \, \erf(\alpha r)/r + Gm \, \erfc(\alpha r)/r$ ($\alpha$ is an
arbitrary constant) and then by applying Poisson summation formula on erfc
terms, gravitational field at position $\vec{r}$  can be written in the form
%
\begin{equation}
\phi (\vec{r})  =  -G \sum_{a=1}^N m_a \left( \sum_{i_1,i_2,i_3} A_S(\vec{r},\vec{r}_a,\vec{l}_{i_1,i_2,i_3}) + 
A_L(\vec{r},\vec{r}_a,\vec{l}_{i_1,i_2,i_3})  \right)
= -G \sum_{a=1}^N m_a f_\mathrm{EF,\Phi}(\vec{r}_a - \vec{r}) \ ,
\label{ewald_sum}
\end{equation}
%
the first sum runs over whole computational domain, where at position
$\vec{r}_a$ is mass $m_a$. Second sum runs over all neigbouring computational
domains, which are at positions $\vec{l}_{i_1,i_2,i_3}$ and
$A_S(\vec{r},\vec{r}_a,\vec{l}_{i_1,i_2,i_3})$ and
$A_L(\vec{r},\vec{r}_a,\vec{l}_{i_1,i_2,i_3})$ are short and long--range
contributions, respectively. It is sufficient to take into account only few
terms in eq. \ref{ewald_sum}. The Ewald field for the acceleration,
$f_\mathrm{EF,a}$, is obtained using a similar decomposition. We modified Ewald
method for problems with periodic boundary conditions in two directions and
isolated boundary conditions in the third direction and for problems with
periodic boundary conditions in one direction and isolated in two directions.

The gravity unit allows also to use a static external gravitational field read
from file. In this version, the field can be either spherically symmetric or
planar being only a function of the z-coordinate. The external field file is a
text file containing three columns of numbers representing the coordinate, the
potential and the acceleration. The coordinate is the radial distance or
z-distance from the center of the external field given by runtime parameters.
The external field if mapped to a grid using a linear interpolation each time
the gravitational acceleration is calculated (in subroutine
\api{physics/Gravity/Gravity\_accelOneRow}).


\section{Usage}
\label{Sec:Gravity Usage}

To include the effects of gravity in your Flash-X executable,
include the option
\begin{quote}
-with-unit=physics/Gravity
\end{quote}
on your command line when you configure the code with \code{setup}.
The default implementation is \code{Constant}, which can be overridden
by including the entire path to the specific implementation in the
command line or \code{Config} file. The other available implementations are
\code{Gravity/\-GravityMain/\-Planepar}, \code{Gravity/\-GravityMain/\-Pointmass} and
\code{Gravity/\-GravityMain/\-Poisson}. The \code {Gravity} unit provides
accessor functions to get gravitational acceleration and
potential. However,
none of the external field implementations of Section~\secref{Sec:GravityExternal}
explicitly
compute the potential, hence they inherit the null implementation from
the API for accessing potential. The gravitation acceleration can be
obtained either on the whole domain, a single block or a single row at
a time.

When building an application that solves the Possion equation for the
gravitational potential, additional storage is needed in \code{unk}
for holding the last, as well as (usually) the previous, gravitational
potential field; and,
depending on the Poisson solver used, additional variables may be needed.
The variables
\code{GPOT\_VAR} and \code{GPOT\_VAR}, and others as needed, will be
automatically defined in \code{Flash.h} in those cases. See
\api{physics/Gravity/Gravity\_potentialListOfBlocks} for more information.


\subsection{Tree Gravity Unit Usage}
\label{Sec:GravityBHTreeUsing}

Calculation of gravitational potential can be enabled by compiling in this unit
and setting the runtime parameter \rpi{Gravity/useGravity} true. The constant of gravity
can be set independently by runtime parameter \rpi{Gravity/grv\_bhNewton}; if it is
not positive, the constant \texttt{Newton} from the Flash-X \unit{PhysicalConstants}
database is used. If parameters \rpi{Grid/gr\_bhPhysMACTW} or \rpi{Grid/gr\_bhPhysMACComm}
are set, the gravity unit MAC is used and it can be chosen by setting
\rpi{Gravity/grv\_bhMAC} to either \texttt{ApproxPartialErr} or \texttt{MaxPartialErr}.
If the first one is used, the order of the multipole approximation is given by
\rpi{Gravity/grv\_bhMPDegree}.

The maximum allowed partial error in gravitational acceleration is set with the
runtime parameter \rpi{Gravity/grv\_bhAccErr}. It has either the meaning of an error in
absolute acceleration or in relative acceleration normalized by the acceleration
from the previous time-step. The latter is used if \rpi{Gravity/grv\_bhUseRelAccErr} is
set to True, and in this case the first call of the tree solver calculates the
potential using purely geometrical MAC (because the acceleration from the
previous time-step does not exist).

Boundary conditions are set by the runtime parameter \rpi{Gravity/grav\_boundary\_type} and they
can be \texttt{isolated}, \texttt{periodic} or \texttt{mixed}.
In the case of mixed boundary conditions, runtime parameters \rpi{Gravity/grav\_boundary\_type\_x},
\rpi{Gravity/grav\_boundary\_type\_y} and \rpi{Gravity/grav\_boundary\_type\_z} specify
along which coordinate boundary conditions are periodic and isolated
(possible values are \texttt{periodic} or \texttt{isolated}).
Arbitrary combination of these values is permitted, thus suitable for problems
with planar resp. linear symmetry. It should work for computational domain with
arbitrary dimensions. The highest accuracy is reached with blocks of cubic physical
dimensions.

If runtime parameter \rpi{Gravity/grav\_boundary\_type} is \texttt{periodic} or
\texttt{mixed}, then the Ewald field for appropriate symmetry is calculated at
the beginning of the simulation. Parameter \rpi{Gravity/grv\_bhEwaldSeriesN} controls the range of indices
$i_1,i_2,i_3$ in (eq. \ref{ewald_sum}). There are two implementations of the
Ewald method: the new one (default) requires less memory and it should be faster
and of comparable accuracy as the old one. The default implementation computes
Ewald field minus the singular $1/r$ term and its partial derivatives on a
single cubic grid, and the Ewald field is then approximated by the first order
Taylor formula. Parameter \rpi{Gravity/grv\_bhEwaldNPer} controls number of grid points
in the $x$ direction in the case of \texttt{periodic} or in periodic
direction(s) in the case of \texttt{mixed} boundary conditions. Since an
elongated computational domain is often desired when \rpi{Gravity/grav\_boundary\_type}
is \texttt{mixed}, the cubic grid would lead to a huge field of data. In this
case, the amount of necessary grid points is reduced by using an analytical
estimate to the Ewald field sufficiently far away of the symmetry plane or axis.

The old implementation (from Flash4.2) is still present and is enabled
by adding
\code{bhtreeEwaldV42=1} on the setup command line.
The Ewald field is then stored
in a nested set of grids, the first of them corresponds in size to full
computational domain, and each following grid is half the size (in each
direction) of the previous grid. Number of nested grids is controlled by
runtime parameter \rpi{Gravity/grv\_bhEwaldNRefV42}. If \rpi{Gravity/grv\_bhEwaldNRefV42} is too low to
cover origin (where is the Ewald field discontinuous), then the run is terminated.
Each grid is composed of \rpi{Gravity/grv\_bhEwaldFieldNxV42} $\times$
\rpi{Gravity/grv\_bhEwaldFieldNyV42} $\times$  \rpi{Gravity/grv\_bhEwaldFieldNzV42} points.
When evaluation of the Ewald Field at particular point is
needed at any time during a run, the field value is found by interpolation in a suitable level of the grid. Linear or
semi-quadratic interpolation can be chosen by runtime parameter
\rpi{Gravity/grv\_bhLinearInterpolOnlyV42} (option \code{true} corresponds to linear
interpolation). Semi-quadratic interpolation is recommended only in the case
when there are periodic boundary conditions in two directions.

The external gravitational field can be switched on by setting
\rpi{Gravity/grv\_useExternalPotential} true. The parameter \rpi{Gravity/grv\_bhExtrnPotFile}
gives the name of the file with the external potential and
\rpi{Gravity/grv\_bhExtrnPotType} specifies the field symmetry: \texttt{spherical} for
the spherical symmetry and \texttt{planez} for the planar symmetry with field
being a function of the z-coordinate. Parameters \rpi{Gravity/grv\_bhExtrnPotCenterY},
\rpi{Gravity/grv\_bhExtrnPotCenterX} and \rpi{Gravity/grv\_bhExtrnPotCenterZ} specify the
position (in the simulation coordinate system) of the external field origin (the
point where the radial or z-coordinate is zero).


\hspace*{-3cm}
\begin{table}[h]
\small
\noindent
\caption{Tree gravity unit parameters controlling the accuracy of calculation.}
\begin{center}
\begin{tabular}{|l|l|l|l|}
\hline
Variable & Type & Default & Description \\
\hline
\rpi{Gravity/grv\_bhNewton}           & real & -1.0 & constant of gravity; if $<$ 0, it is obtained\\
                                      &      &      & from internal physical constants database\\
\hline
\rpi{Gravity/grv\_bhMAC}              & string & "ApproxPartialErr" & MAC, other option: "MaxPartialErr"\\
\hline
\rpi{Gravity/grv\_bhMPDegree}         & integer & 2 & degree of multipole in error estimate in APE MAC \\
\hline
\rpi{Gravity/grv\_bhUseRelAccErr}     & logical & .false. & if .true., grv\_bhAccErr has meaning of\\
                                      &         &         & relative error, otherwise absolute \\
\hline
\rpi{Gravity/grv\_bhAccErr}           & real & 0.1 & maximum allowed error in gravitational\\
                                      &      &     & acceleration \\
\hline
\end{tabular}
\end{center}
\end{table}


\hspace*{-3cm}
\begin{table}[h]
\small
\noindent
\caption{Tree gravity unit parameters controlling the boundary conditions.}
\begin{center}
\begin{tabular}{|l|l|l|l|}
\hline
Variable & Type & Default & Description \\
\hline
\rpi{Gravity/grav\_boundary\_type}     & string & "isolated" & or "periodic" or "mixed"\\
\hline
\rpi{Gravity/grav\_boundary\_type\_x}   & string & "isolated" & or "periodic"\\
\hline
\rpi{Gravity/grav\_boundary\_type\_y}   & string & "isolated" & or "periodic"\\
\hline
\rpi{Gravity/grav\_boundary\_type\_z}   & string & "isolated" & or "periodic"\\
\hline
\rpi{Gravity/grv\_bhEwaldAlwaysGenerate} & boolean & true & whether Ewald field should be regenerated\\
%instead of read from file \\
\hline
\rpi{Gravity/grv\_bhEwaldSeriesN} & integer & $10$ & number of terms in the Ewald series\\
\hline
\rpi{Gravity/grv\_bhEwaldNPer}  & integer & 32 & number of points+1 of the Taylor expansion \\
\hline
\rpi{Gravity/grv\_bhEwaldFName} & string  & "ewald\_coeffs" & file with coefficients of the Ewald field Taylor expansion\\
\hline
\hline
\rpi{Gravity/grv\_bhEwaldFieldNxV42} & integer & $32$ & size of the Ewald field grid in x-direction\\
\hline
\rpi{Gravity/grv\_bhEwaldFieldNyV42} & integer & $32$ & size of the Ewald field grid in y-direction\\
\hline
\rpi{Gravity/grv\_bhEwaldFieldNzV42} & integer & $32$ & size of the Ewald field grid in z-direction\\
\hline
\rpi{Gravity/grv\_bhEwaldNRefV42}    & integer & -1 & number of refinement levels (nested grids) for the Ewald\\
                                      &         &         & field; if $<$ 0, determined automatically\\
\hline
\rpi{Gravity/grv\_bhLinearInterpolOnlyV42} & logical & .true. & if .false., semi-quadratic interpolation is used for\\
                                      &         &         & interpolation in the Ewald field\\
\hline
\rpi{Gravity/grv\_bhEwaldFNameAccV42} & string  & "ewald\_field\_acc" & file with the Ewald field for acceleration\\
\hline
\rpi{Gravity/grv\_bhEwaldFNamePotV42} & string  & "ewald\_field\_pot" & file with coefficients of the Ewald field for potential\\
\hline
\end{tabular}
\end{center}
\end{table}


Tree gravity unit parameters controlling the external gravitational field.

\hspace*{-3cm}
\small
\noindent
\begin{center}
\begin{tabular}{|l|l|l|l|}
\hline
Variable & Type & Default & Description \\
\hline
\rpi{Gravity/grv\_bhUseExternalPotential} & logical & .false. & whether to use external field \\
\hline
\rpi{Gravity/grv\_bhUsePoissonPotential}  & logical & .true. & whether to use gravitational field calculated by the\\
                                      &         &         & tree solver \\
\hline
\rpi{Gravity/grv\_bhExtrnPotFile}  & string & "external\_potential.dat" & file containing the external gravitational field \\
\hline
\rpi{Gravity/grv\_bhExtrnPotType}  & string & "planez" & type of the external field: planar or spherical symmetry\\
\hline
\rpi{Gravity/grv\_bhExtrnPotCenterX} & real & 0.0 & x-coordinate of the center of the external field \\
\hline
\rpi{Gravity/grv\_bhExtrnPotCenterY} & real & 0.0 & y-coordinate of the center of the external field \\
\hline
\rpi{Gravity/grv\_bhExtrnPotCenterZ} & real & 0.0 & z-coordinate of the center of the external field \\
\hline
\end{tabular}
\end{center}
\normalsize


\section{Unit Tests}
\label{Sec:GravityUnitTests}
There are two unit tests for the gravity unit.  \code{Poisson3} is essentially
the Maclaurin spheroid problem described in \secref{Sec:SimulationMacLaurin}.
Because an analytical solution exists, the accuracy of the gravitational solver
can be quantified.  The second test, \code{Poisson3_active} is a modification
of \code{Poisson3} to test the mapping of particles in
\api{Grid/Grid_mapParticlesToMesh}.  Some of the mesh density is redistributed
onto particles, and the particles are then mapped back to the mesh, using
the analytical solution to verify completeness.  This test is similar to
the simulation \code{PoisParticles} discussed in \secref{Sec:SimulationPoisParticles}.
\code{PoisParticles} is based on the Huang-Greengard Poisson gravity test described
in \secref{Sec:SimulationPoisTest}.
