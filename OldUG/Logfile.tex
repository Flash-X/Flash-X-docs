\chapter{Logfile Unit}\label{Chp:Logfile Unit}



Flash-X supplies the \unit{Logfile} unit to manage an output log during a Flash-X simulation.
The logfile contains various types of useful information, warnings,
and error messages produced by a Flash-X run.
Other units can add information to  the logfile through the \unit{Logfile} unit interface.
The \code{Logfile} routines enable a program to open and close a log file, write time or date
stamps to the file, and write arbitrary messages to the file.
The file is kept closed and is only opened for appending
when information is to be written, thus avoiding problems with
unflushed buffers.  For this reason, \code{Logfile} routines should
not be called within time-sensitive loops, as system calls are generated.
Even when starting from scratch, the logfile is opened in append
mode to avoid deleting important logfiles. Two kinds of Logfiles are
supported. The first kind is similar to that in Flash-X2 and early releases of
\flashx, where the master processor has exclusive access to the
logfile and writes global information to it. The newer kind gives all
processors access to their own private logfiles if they need to have
one. Similar to the traditional logfile, the private logfiles are
opened in append mode, and they are created the first time a
processor writes to one. The private logfiles are extremely useful to
gather information about failures causes by a small fraction of
processors; something that cannot be done in the traditional logfile.

The \unit{Logfile} unit is included by default in all the provided Flash-X
simulations because it is required by the \code{Driver/\-DriverMain} \code{Config}.
As with all the other units in Flash-X, the data specific to the Logfile
unit is stored in the module \code{Logfile\_data.F90}.  Logfile
unit scope data variables begin with the prefix \code{log\_variableName} and they are
initialized in the routine \api{monitors/Logfile/Logfile_init}.

By default, the logfile is named \code{flash.log} and found in the output directory.
The user may change the name of the logfile by altering the runtime parameter \rpi{Logfile/log_file} in the
\code{flash.par}.
\begin{codeseg}
# names of files
basenm   = "cellular_"
log_file = "cellular.log"
\end{codeseg}

\section{Meta Data}
The
\code{logfile} stores meta data about a given run including the time
and date of the run, the number of MPI tasks, dimensionality,
compiler flags and other information about the run.  The snippet below
is an example from a \code{logfile} showing the basic setup and compilation
information:

\begin{codeseg}
================================================================================
 Number of MPI tasks:                   2
 MPI version:                           1
 MPI subversion:                        2
 Dimensionality:                        2
 Max Number of Blocks/Proc:          1000
 Number x zones:                        8
 Number y zones:                        8
 Number z zones:                        1
 Setup stamp:     Wed Apr 19 13:49:36 2006
 Build stamp:     Wed Apr 19 16:35:57 2006
 System info:
 Linux zingiber.uchicago.edu 2.6.12-1.1376_FC3smp #1 SMP Fri Aug 26 23:50:33 EDT
 Version:         Flash-X 3.0.
 Build directory: /home/kantypas/Flash-X3/trunk/Sod
 Setup syntax:
 /home/kantypas/Flash-X3/trunk/bin/setup.py Sod -2d -auto -unit=IO/IOMain/hdf5/parallel/PM
                    -objdir=Sod
 f compiler flags:
 /usr/local/pgi6/bin/pgf90 -I/usr/local/mpich-pg/include -c -r8 -i4 -fast -g
                    -DMAXBLOCKS=1000 -DNXB=8 -DNYB=8 -DNZB=1 -DN_DIM=2
 c compiler flags:
 /usr/local/pgi6/bin/pgcc -I/usr/local/hdf5-pg/include -I/usr/local/mpich-pg/include
                    -c -O2 -DMAXBLOCKS=1000 -DNXB=8 -DNYB=8 -DNZB=1 -DN_DIM=2
===============================================================================
\end{codeseg}

\section{Runtime Parameters, Physical Constants, and Multispecies Data}
The \code{logfile} also records which units were included in a
simulation, the runtime parameters, physical constants, and any species
and their properties from the \unit{Multispecies} unit.  The \flashx logfile
keeps track of whether a runtime parameter is a default value or
whether its value has been redefined in the \code{flash.par}%\index{flash.par@\code{flash.par}} file.
The
\code{[CHANGED]} symbol will occur next to a runtime parameter if its
value has been redefined in the \code{flash.par}.
 Note that the runtime parameters are output in alphabetical order within the Fortran
datatype -- so integer parameters are shown first, then real, then string, then Boolean.
The snippet below shows the this portion of the \code{logfile}; omitted sections are indicated with
``...".

\begin{codeseg}
 ==============================================================================
  Flash-X Units used:
   Driver
   Driver/DriverMain
   Driver/DriverMain/TimeDep
   Grid
   Grid/GridMain
   Grid/GridMain/paramesh
   Grid/GridMain/paramesh/paramesh4
   ...
   Multispecies
   Particles
   PhysicalConstants
   PhysicalConstants/PhysicalConstantsMain
   RuntimeParameters
   RuntimeParameters/RuntimeParametersMain
   ...
   physics/utilities/solvers/LinearAlgebra
 ==============================================================================
 RuntimeParameters:

 ==============================================================================
algebra                     =          2 [CHANGED]
bndpriorityone              =          1
bndprioritythree            =          3
...
cfl                         =                 0.800E+00
checkpointfileintervaltime  =                 0.100E-08 [CHANGED]
cvisc                       =                 0.100E+00
derefine_cutoff_1           =                 0.200E+00
derefine_cutoff_2           =                 0.200E+00
...
zmax                        =                 0.128E+02 [CHANGED]
zmin                        =                 0.000E+00
basenm                      = cellular_                      [CHANGED]
eosmode                     = dens_ie
eosmodeinit                 = dens_ie
geometry                    = cartesian
log_file                    = cellular.log                   [CHANGED]
output_directory            =
pc_unitsbase                = CGS
plot_grid_var_1             = none
plot_grid_var_10            = none
plot_grid_var_11            = none
plot_grid_var_12            = none
plot_grid_var_2             = none
...
yr_boundary_type            = periodic
zl_boundary_type            = periodic
zr_boundary_type            = periodic
bytepack                    =  F
chkguardcells               =  F
converttoconsvdformeshcalls =  F
converttoconsvdinmeshinterp =  F
...
useburn                     =  T [CHANGED]
useburntable                =  F

 ==============================================================================

 Known units of measurement:

              Unit                          CGS Value                Base Unit
  1                  cm                     1.0000                           cm
  2                   s                     1.0000                            s
  3                   K                     1.0000                            K
  4                   g                     1.0000                            g
  5                 esu                     1.0000                          esu
  6                   m                     100.00                           cm
  7                  km                    0.10000E+06                       cm
  8                  pc                    0.30857E+19                       cm
  ...
 Known physical constants:

    Constant Name       Constant Value   cm        s         g         K         esu
  1              Newton    0.66726E-07   3.00     -2.00     -1.00      0.00      0.00
  2      speed of light    0.29979E+11   1.00     -1.00      0.00      0.00      0.00
 ...
 15               Euler    0.57722       0.00      0.00      0.00      0.00      0.00
 ==============================================================================

 Multifluid database contents:

Initially defined values of species:
Name     Index          Total   Positive  Neutral   Negative  bind Ener Gamma
ar36        12       3.60E+01  1.80E+01 -9.99E+02 -9.99E+02  3.07E+02 -9.99E+02
c12         13       1.20E+01  6.00E+00 -9.99E+02 -9.99E+02  9.22E+01 -9.99E+02
ca40        14       4.00E+01  2.00E+01 -9.99E+02 -9.99E+02  3.42E+02 -9.99E+02
...
ti44        24       4.40E+01  2.20E+01 -9.99E+02 -9.99E+02  3.75E+02 -9.99E+02
 ==============================================================================
\end{codeseg}

\section{Accessor Functions and Timestep Data}
Other units within Flash-X may make calls to write information, or
stamp, the logfile.  For example, the \unit{Driver} unit calls the API
routine \api{monitors/Logfile/Logfile_stamp} after each timestep.  The
\unit{Grid} unit calls \api{monitors/Logfile/Logfile_stamp} whenever
refinement occurs in an adaptive grid simulation.  If there is an
error that is caught in the code the API routine
\api{Driver/Driver_abortFlash} stamps the \code{logfile} before
aborting the code.  Any unit can stamp the logfile with one of two
routines \api{monitors/Logfile/Logfile_stamp} which includes a data
and time stamp along with a logfile message, or
\api{monitors/Logfile/Logfile_stampMessage} which simply writes a
string to the \code{logfile}.

The routine \api{monitors/Logfile/Logfile_stamp} is overloaded so the user must
use the interface file \code{Logfile_interface.F90} in the calling routine.  The next
snippit shows logfile output during the evolution loop of a Flash-X run.

\begin{codeseg}
 ==============================================================================
 [ 04-19-2006  16:40.43 ] [Simulation_init]: initializing Sod problem
 [GRID amr_refine_derefine]             initiating refinement
 [GRID amr_refine_derefine] min blks 0    max blks 1    tot blks 1
 [GRID amr_refine_derefine] min leaf blks 0    max leaf blks 1    tot leaf blks 1
 [GRID amr_refine_derefine]             refinement complete
 [ 04-19-2006  16:40.43 ] [GRID gr_expandDomain]: create level=2
 ...
 [GRID amr_refine_derefine]             initiating refinement
 [GRID amr_refine_derefine] min blks 250    max blks 251    tot blks 501
 [GRID amr_refine_derefine] min leaf blks 188    max leaf blks 188    tot leaf blks 376
 [GRID amr_refine_derefine]             refinement complete
 [ 04-19-2006  16:40.44 ] [GRID gr_expandDomain]: create level=7
 [ 04-19-2006  16:40.44 ] [GRID gr_expandDomain]: create level=7
 [ 04-19-2006  16:40.44 ] [GRID gr_expandDomain]: create level=7
 [ 04-19-2006  16:40.44 ] [IO_writeCheckpoint] open: type=checkpoint name=sod_hdf5_chk_0000
 [ 04-19-2006  16:40.44 ] [io_writeData]: wrote     501          blocks
 [ 04-19-2006  16:40.44 ] [IO_writeCheckpoint] close: type=checkpoint name=sod_hdf5_chk_0000
 [ 04-19-2006  16:40.44 ] [IO writePlotfile] open: type=plotfile name=sod_hdf5_plt_cnt_0000
 [ 04-19-2006  16:40.44 ] [io_writeData]: wrote     501          blocks
 [ 04-19-2006  16:40.44 ] [IO_writePlotfile] close: type=plotfile name=sod_hdf5_plt_cnt_0000
 [ 04-19-2006  16:40.44 ] [Driver_evolveFlash]: Entering evolution loop
 [ 04-19-2006  16:40.44 ] step: n=1 t=0.000000E+00 dt=1.000000E-10
 ...
 [ 04-19-2006  16:41.06 ] [io_writeData]: wrote     501          blocks
 [ 04-19-2006  16:41.06 ] [IO_writePlotfile] close: type=plotfile name=sod_hdf5_plt_cnt_0002
 [ 04-19-2006  16:41.06 ] [Driver_evolveFlash]: Exiting evolution loop
 ==============================================================================
\end{codeseg}

\section{Performance Data}
\label{Sec:LogfilePerformance}
Finally, the \code{logfile} records performance data for the
simulation.  The \unit{Timers} unit (see \secref{Sec:Timers Unit}) is
responsible for storing, collecting and interpreting the performance
data.  The \unit{Timers} unit calls the API routine
\api{monitors/Logfile/Logfile_writeSummary} to format the performance data and
write it to the logfile.  The snippet below shows the
performance data section of a logfile.

\begin{codeseg}
 ==============================================================================
 perf_summary: code performance summary
                      beginning : 04-19-2006  16:40.43
                         ending : 04-19-2006  16:41.06
   seconds in monitoring period :               23.188
         number of subintervals :                   21
        number of evolved zones :                16064
               zones per second :              692.758
 ------------------------------------------------------------------------------
 accounting unit                       time sec  num calls   secs avg  time pct
 ------------------------------------------------------------------------------
 initialization                          1.012      1           1.012     4.366
  guardcell internal                     0.155     17           0.009     0.669
  writeCheckpoint                        0.085      1           0.085     0.365
  writePlotfile                          0.061      1           0.061     0.264
 evolution                              22.176      1          22.176    95.633
  hydro                                 18.214     40           0.455    78.549
   guardcell internal                    2.603     80           0.033    11.227
  sourceTerms                            0.000     40           0.000     0.002
  particles                              0.000     40           0.000     0.001
  Grid_updateRefinement                  1.238     20           0.062     5.340
   tree                                  1.126     10           0.113     4.856
    guardcell tree                       0.338     10           0.034     1.459
     guardcell internal                  0.338     10           0.034     1.458
    markRefineDerefine                   0.339     10           0.034     1.460
     guardcell internal                  0.053     10           0.005     0.230
    amr_refine_derefine                  0.003     10           0.000     0.011
    updateData                           0.002     10           0.000     0.009
    guardcell                            0.337     10           0.034     1.453
     guardcell internal                  0.337     10           0.034     1.452
   eos                                   0.111     10           0.011     0.481
   update particle refinemen             0.000     10           0.000     0.000
  io                                     2.668     20           0.133    11.507
   writeCheckpoint                       0.201      2           0.101     0.868
   writePlotfile                         0.079      2           0.039     0.340
   diagnostics                           0.040     20           0.002     0.173
 ==============================================================================
 [ 04-19-2006  16:41.06 ] LOGFILE_END: Flash-X run complete.
\end{codeseg}





\section{Example Usage}

An example program using the Logfile unit might appear as follows:

\begin{codeseg}
program testLogfile

    use Logfile_interface, ONLY: Logfile_init, Logfile_stamp, Logfile_open, Logfile_close
    use Driver_interface, ONLY: Driver_initParallel
    use RuntimeParameters_interface, ONLY: RuntimeParameters_init
    use PhysicalConstants_interface, ONLY: PhysicalConstants_init

    implicit none

    integer :: i
    integer :: log_lun 
    integer :: myPE, numProcs
    logical :: restart, localWrite

    call Driver_initParallel(myPE, numProcs) !will initialize MPI
    call RuntimeParameters_init(myPE, restart) ! Logfile_init needs runtime parameters
    call PhysicalConstants_init(myPE) ! PhysicalConstants information adds to logfile
    call Logfile_init(myPE, numProcs) ! will end with Logfile_create(myPE, numProcs)

    call Logfile_stamp (myPE, "beginning log file test...", "[programtestLogfile]")
    localWrite=.true.
    call Logfile_open(log_lun,localWrite) !! open the local logfile
    do i = 1, 10
      write (log_lun,*) 'i = ', i
    enddo
    call Logfile_stamp (myPE, "finished logfile test", "[program testLogfile]")
    call Logfile_close(myPE, log_lun)
  

end program testLogfile
\end{codeseg}

