\chapter{Equation of State Unit}
\label{Chp:EOS Unit}

\section{Introduction}
\label{Sec:Eos Intro}
The \code{Eos} unit implements the equation
of state needed by the hydrodynamics and nuclear burning solvers.
The function \api{physics/Eos/Eos}
provides the interface for
operating on a
one-dimensional vector. The same interface can be used for a single
cell by reducing the vector size to 1. Additionally, this function
can be used to find the thermodynamic quantities either from the
density, temperature, and composition or from the density, internal
energy, and composition. For user's convenience,
a wrapper function (\api{physics/Eos/Eos_wrapped}) is provided, which
takes a section of a block and translates it into the data format required
by the \api{physics/Eos/Eos} function, then calls the function. Upon
return from the \api{physics/Eos/Eos} function, the wrapper translates the
returned data back to the same section of the block.

Four implementations of the (\code{Eos}) unit are available in the
current release of \flashx: \texttt{Gamma}
% \index{equation of state!gamma-law},
which implements a perfect-gas equation of state;
{\tt Gamma/RHD}%\index{equation of state!rhd-gamma}, 
which implements a perfect-gas equation taking relativistic effects into account; 
\code{Multigamma}%\index{equation of state!multi-gamma},
which implements a perfect-gas equation of state
with multiple fluids, each of which can have its own adiabatic index
($\gamma$); and \code{Helmholtz}%\index{equation of state!Helmholtz},
which uses a fast Helmholtz free-energy table interpolation to handle
degenerate/relativistic electrons/positrons and includes
radiation pressure and ions (via the perfect gas approximation).

As described in previous sections,
Flash-X evolves the Euler equations for
compressible, inviscid flow. This system of equations must be closed
by an additional equation that provides a relation between the
thermodynamic quantities of the gas.  This relationship
is known as the equation of
state for the material, and its structure and properties depend on the
composition of the gas.

It is common to call an equation of state (henceforth EOS) routine more
than $10^9$ times during a two-dimensional simulation and more than
$10^{11}$ times during the course of a three-dimensional simulation of
stellar phenomena.  Thus, it is very desirable to have an EOS
that is as efficient as possible, yet accurately represents the
relevant physics. While Flash-X is capable of using any
general equation of state, we discuss here the
three equation of state routines that are supplied: an ideal-gas or gamma-law
EOS, an EOS for a fluid composed of multiple gamma-law gases, and a
tabular Helmholtz free energy EOS appropriate for stellar
interiors. The two gamma-law EOSs consist of simple analytic expressions
that make for a very fast EOS routine both in the case of a single gas
or for a mixture of gases. The Helmholtz EOS includes much more
physics and relies on a table look-up scheme for performance.

\section{Gamma Law and Multigamma}
\label{Sec:Eos Gammas}
Flash-X uses the method of Colella \& Glaz (1985) to handle general
equations of state.  General equations of state contain 4 adiabatic
indices (Chandrasekhar 1939), but the method of Colella \& Glaz
parameterizes the EOS and requires only two of the adiabatic
indices%\index{adiabatic indices}.
The first is necessary to calculate
the adiabatic sound speed and is given by
\begin{equation}
\gamma_1 = \frac{\rho}{P}\frac{\partial P}{\partial \rho} \; .
\end{equation}
The second relates the pressure to the energy and is given by
\begin{equation}
\label{Eqn:game}\gamma_4 = 1 + \frac{P}{\rho\epsilon} \; .
\end{equation}
These two adiabatic indices are stored as the mesh-based variables \code{GAMC_VAR} and
\code{GAME_VAR}.
All EOS routines must return $\gamma_1$, and $\gamma_4$ is calculated from
\eqref{Eqn:game}.

The gamma-law%\index{equation of state!gamma-law}
EOS models a simple
ideal gas with a constant adiabatic index $\gamma$. Here we have
dropped the subscript on $\gamma$, because for an ideal gas, all
adiabatic indices are equal.  The relationship between pressure $P$,
density $\rho$, and specific internal energy $\epsilon$ is
\begin{equation}
\label {Eqn:eos2b}
%{P = {N_a \ k \over \bar{A}} \rho T}
P = \left(\gamma - 1\right)\rho\epsilon~.
%\epsilon = {1 \over \gamma - 1} \ {P \over \rho}
\end{equation}
We also have an expression relating pressure to the temperature $T$
\begin{equation}
\label {Eqn:eos2a}
P = \frac{N_a k}{\bar{A}} \rho T ~,
\end{equation}
where $N_a$ is the Avogadro number, $k$ is the Boltzmann
constant, and $\bar{A}$ is the average atomic mass, defined as
\begin{equation}
\frac{1}{\bar{A}} = \sum_{i}\frac{X_{i}}{A_{i}}~,
\end{equation}
where $X_i$ is the mass fraction of the $i$th element.
Equating these expressions for pressure yields an expression for the
specific internal energy as a function of temperature
\begin{equation}
\epsilon = \frac{1}{\gamma - 1} \frac{N_a k} {\bar{A}} T~.
\end{equation}
The relativistic variant of the ideal gas equation is explained in more detail
in \secref{Sec:RHD}.


Simulations are not restricted to a single ideal gas; the multigamma
EOS%\index{equation of state!multi-gamma} provides routines for
simulations with several species of ideal gases each with its own
value of $\gamma$. In this case the above expressions hold, but
$\gamma$ represents the weighted average adiabatic index calculated
from
\begin{equation}
\frac{1}{\left(\gamma - 1\right)} = \bar{A}\sum_{i}\frac{1}{\left(\gamma_{i} -
1\right)}\frac{X_{i}}{A_{i}}~.
\end{equation}

We note that the analytic expressions apply to both the forward
(internal energy as a function of density, temperature, and
composition) and backward (temperature as a function of density,
internal energy and composition) relations.  Because the backward
relation requires no iteration in order to obtain the temperature,
this EOS is quite inexpensive to evaluate.  Despite its fast performance,
use of the gamma-law EOS is limited, due to its restricted range of
applicability for astrophysical problems.


\section{Helmholtz}
\label{Sec:Eos Helmholtz}

The Helmholtz EOS%\index{equation of state!Helmholtz}
provided with the
Flash-X distribution contains more physics and is appropriate for
addressing astrophysical phenomena in which electrons and positrons
may be relativistic and/or degenerate and in which radiation may
significantly contribute to the thermodynamic state.
Full details of the Helmholtz equation of state are provided in Timmes \& Swesty
(1999).
This EOS includes
contributions from radiation, completely ionized nuclei, and
degenerate/relativistic electrons and positrons.  The pressure and
internal energy are calculated as the sum over the components
\begin{equation}
\label {Eqn:eos3a}
P_{\rm tot} = P_{\rm rad} + P_{\rm ion} + P_{\rm ele} + P_{\rm pos} + P_{\rm
coul}
\end{equation}
\begin{equation}
\label {Eqn:eos3b}
\epsilon_{\rm tot} = \epsilon_{\rm rad} + \epsilon_{\rm ion} +
\epsilon_{\rm ele} + \epsilon_{\rm pos}  + \epsilon_{\rm coul} \; .
\end{equation}
Here the subscripts ``rad,'' ``ion,'' ``ele,'' ``pos,'' and ``coul'' represent
the contributions from radiation, nuclei, electrons, positrons, and
corrections for Coulomb effects, respectively.
The radiation portion assumes a blackbody in local thermodynamic
equilibrium, the ion portion (nuclei) is treated as an ideal gas
with $\gamma \, = \, 5/3$, and the electrons and positrons are treated as a
non-interacting Fermi gas.

The blackbody pressure and energy are calculated as
\begin{equation}
\label {Eqn:eos4a}
P_{\rm rad} = {a T^4 \over 3}
\end{equation}
\begin{equation}
\label {Eqn:eos4b}
\epsilon_{\rm rad} = { 3 P_{\rm rad} \over \rho} \,
\end{equation}
where $a$ is related to the Stephan-Boltzmann constant $\sigma_B \, = \,
a c/4$, and $c$ is the speed of light. The ion portion of each routine
is the ideal gas of (\eqnsref{Eqn:eos2b}{Eqn:eos2a}) with $\gamma \, =
\, 5/3$. The number densities of free electrons $N_{\rm ele}$ and
positrons $N_{\rm pos}$ in the noninteracting Fermi gas formalism are
given by
\begin{equation}
\label {Eqn:eos6a}
N_{\rm ele} = {8 \pi \sqrt{2} \over h^3} \
                  m_{\rm e}^3 \ c^3 \ \beta^{3/2} \
          \left[ F_{1/2}(\eta,\beta) \ + \ F_{3/2}(\eta,\beta) \right]
\end{equation}
\begin{equation}
\label {Eqn:eos6b}
N_{\rm pos} = {8 \pi \sqrt{2} \over h^3} \
                  m_{\rm e}^3 \ c^3 \ \beta^{3/2}
  \left[
         F_{1/2} \left( -\eta - 2/\beta, \beta \right)
         \ + \ \beta \
         F_{3/2} \left( -\eta - 2 /\beta, \beta \right)
    \right] \enskip ,
\end{equation}
where $h$ is Planck's constant, $m_{\rm e}$ is the electron rest mass,
$\beta \: = \: k T / (m_{\rm e} c^2)$ is the relativity parameter,
$\eta \: = \: \mu / k T$ is the normalized chemical potential energy
$\mu$ for electrons, and $F_{k}(\eta,\beta)$ is the
\hbox{Fermi-Dirac} integral
\begin{equation}
\label{Eqn:eos7}
    F_{k}(\eta,\beta) = \int\limits_{0}^{\infty} \
    {x^{k} \ (1 + 0.5 \ \beta \ x)^{1/2} \ dx
     \over
     \exp(x - \eta) + 1
    }\ .
\end{equation}
Because the electron rest mass is not included in the chemical
potential, the positron chemical potential must have the form
$\eta_{{\rm pos}} \, = \, -\eta - 2/\beta$.  For complete ionization,
the number density of free electrons in the matter is
\begin{equation}
\label{Eqn:eos8}
N_{\rm ele,matter}
   = {\bar{Z} \over \bar{A}} \ N_a \ \rho = \bar{Z} \ N_{\rm ion}
\ ,
\end{equation}
and charge neutrality requires
\begin{equation}
\label{Eqn:eos9}
N_{\rm ele,matter} = N_{\rm ele} - N_{\rm pos} \ .
\end{equation}
Solving this equation with a standard one-dimensional root-finding
algorithm determines $\eta$.  Once $\eta$ is known, the Fermi-Dirac
integrals can be evaluated, giving the pressure, specific thermal
energy, and entropy due to the free electrons and positrons. From
these, other thermodynamic quantities such as $\gamma_1$ and
$\gamma_4$ are found. Full details of this formalism may be found in
Fryxell {\it et al.} (2000) and references therein.

The above formalism requires many complex calculations to evaluate the
thermodynamic quantities, and routines for these calculations
typically are designed for accuracy and thermodynamic consistency at
the expense of speed. The Helmholtz EOS in Flash-X provides a table of
the Helmholtz free energy (hence the name) and makes use of a
thermodynamically consistent interpolation scheme
obviating the need to perform the complex
calculations required of the above formalism during the course of a
simulation. The interpolation scheme uses a bi-quintic Hermite
interpolant resulting in an accurate EOS that performs
reasonably well.

The Helmholtz free energy,
\begin{equation}
\label {Eqn:eos13a}
F = \epsilon - T \ S
\end{equation}
\begin{equation}
\label {Eqn:eos13b}
dF = -S \ dT + {P \over \rho^2} \ d\rho
\enskip ,
\end{equation}
is the appropriate thermodynamic potential for use when the
temperature and density are the natural thermodynamic variables.  The
free energy table distributed with Flash-X was produced from the Timmes
EOS (Timmes \& Arnett 1999). The Timmes EOS evaluates the Fermi-Dirac
integrals \eqref{Eqn:eos7} and their partial derivatives with
respect to $\eta$ and $\beta$ to machine precision with the efficient
quadrature schemes of Aparicio (1998) and uses a Newton-Raphson
iteration to obtain the chemical potential of \eqref{Eqn:eos9}.
All partial derivatives of the pressure, entropy, and internal energy
are formed analytically.  Searches through the free energy table are
avoided by computing hash indices from the values of any given
$(T,\rho \bar{Z}/\bar{A})$ pair.  No computationally expensive
divisions are required in interpolating from the table; all of them
can be computed and stored the first time the EOS routine is called.

We note that the Helmholtz free energy table is constructed for only
the electron-positron plasma, and it is a 2-dimensional function of
density and temperature, {\it i.e.} $F(\rho,{\rm T})$. It is made with
${\bar {\rm A}} \, = \, {\bar {\rm Z}} = 1$ (pure hydrogen), with an
electron fraction $Y_{\rm e} \, = \, 1$.  One
reason for not including contributions from photons and ions in the
table is that these components of the Helmholtz EOS are very simple
(\eqnsref{Eqn:eos4a}{Eqn:eos4b}), and one doesn't need fancy table
look-up schemes to evaluate simple analytical functions.  A more
important reason for only constructing an electron-positron EOS table
with $Y_{\rm e} \, = \, 1$ is that the 2-dimensional table is valid
for {\it any\/} composition. Separate planes for each $Y_{\rm e}$ are
not necessary (or desirable), since simple multiplication by $Y_{\rm
e}$ in the appropriate places gives the desired composition
scaling. If photons and ions were included in the table, then this
valuable composition independence would be lost, and a 3-dimensional
table would be necessary.

The Helmholtz EOS has been subjected to considerable analysis and
testing (Timmes \& Swesty 2000), and particular care was taken to
reduce the numerical error introduced by the thermodynamical models
below the formal accuracy of the hydrodynamics algorithm (Fryxell,
et al. 2000; Timmes \& Swesty 2000).  The physical limits of the
Helmholtz EOS are $10^{-10}\,<\,\rho\,<\,10^{11}~({\rm g~cm}^{-3})$
and $10^{4}\,<\,T\,<\,10^{11}$ (K).  As with the gamma-law EOS, the
Helmholtz EOS provides both forward and backward relations. In the
case of the forward relation ($\rho, T$, given along with the
composition) the table lookup scheme and analytic formulae directly
provide relevant thermodynamic quantities. In the case of the
backward relation ($\rho, \epsilon$, and composition given), the
routine performs a Newton-Rhaphson iteration to determine
temperature.  It is possible for the input variables to be
changed in the iterative modes since the solution is not exact.
The returned quantities are thermodynamically consistent.


\section{Usage}
\label{Sec:Eos Usage}

\subsection{Initialization}
\label{Sec:Eos Initialization}
The initialization function of the Eos unit
\api{physics/Eos/Eos_init} is fairly simple for the two ideal gas
gamma law implementations included. It gathers the runtime parameters
and the physical constants needed by the equation of state and stores
them in the data module.
The Helmholtz EOS \api{physics/Eos/Eos_init} routine is a little more
complex. The \code{Helmholtz} EOS requires an input file
\code{helm\_table.dat} that contains the lookup table for the electron
contributions.  This table is currently stored in ASCII for portability
purposes.  When the table is first read in, a binary version called
\code{helm\_table.bdat} is created.  This binary format can be used for
faster subsequent
restarts on the same machine but may not be portable across
platforms. The \code{Eos_init} routine reads in the table data on processor 0 and
broadcasts it to all other processors.

\subsection{Runtime Parameters}
\label{Sec:Eos Runtime Parameters}
Runtime parameters for the \code{Gamma} unit require the user to set the
thermodynamic properties for the single gas.  \rpi{Eos/gamma},
\rpi{Eos/eos_singleSpeciesA}, \rpi{Eos/eos_singleSpeciesZ} set the ratio of specific
heats and the nucleon and proton numbers for the gas.  In contrast, the
\code{Multigamma} implementation does not set runtime parameters to define
properties of the multiple species.  Instead, the simulation \code{Config} file
indicates the requested species, for example helium and oxygen can be defined as
\begin{center}
\begin{verbatim}
SPECIES HE4
SPECIES O16
\end{verbatim}
\end{center}
The properties of the gases are initialized in the file \api{Simulation/Simulation_initSpecies}\code{.F90},
for example
\begin{center}
\begin{verbatim}
subroutine Simulation_initSpecies()
  use Multispecies_interface, ONLY : Multispecies_setProperty
  implicit none
#include "Flash.h"
#include "Multispecies.h"
  call Multispecies_setProperty(HE4_SPEC, A, 4.)
  call Multispecies_setProperty(HE4_SPEC, Z, 2.)
  call Multispecies_setProperty(HE4_SPEC, GAMMA, 1.66666666667e0)
  call Multispecies_setProperty(O16_SPEC, A, 16.0)
  call Multispecies_setProperty(O16_SPEC, Z, 8.0)
  call Multispecies_setProperty(O16_SPEC, GAMMA, 1.4)
end subroutine Simulation_initSpecies
\end{verbatim}
\end{center}


For the Helmholtz equation of state, the table-lookup algorithm requires
a given temperature and density. When temperature or internal energy are supplied
as the input parameter, an iterative solution is found.
Therefore, no matter what mode is selected for \code{Helmholtz} input, the best
 initial value of temperature should be provided to speed convergence of the iterations.
The iterative solver is controlled by two runtime parameters \rpi{Eos/eos_maxNewton} and
\rpi{Eos/eos_tolerance} which define the maximum number of iterations
and convergence tolerance.
An additional runtime parameter for \code{Helmholtz}, \rpi{Eos/eos_coulumbMult},
indicates whether or not to apply Coulomb corrections. In some
regions of the $\rho$-$T$ plane, the approximations made in the
Coulomb corrections may be invalid and result in negative pressures.
When the parameter \code{eos_coulombMult} is set to zero,
the Coulomb corrections are not applied.

\subsection{Direct and Wrapped Calls}
\label{Sec:Eos Wrapper}
The primary function in the \unit{Eos} unit%\index{equation of state!Eos@\code{Eos}}, \api{physics/Eos/Eos},
operates on a vector, taking density, composition, and either
temperature, internal energy, or pressure as input, and returning
$\gamma_1$, and either the pressure, temperature or internal energy
(whichever was not used as input).
This equation of state interface is useful for initializing a
problem.  The user is given direct control over the input and output,
since everything is passed through the argument list. Also, the vector
data format is more efficient than calling the
equation of state routine directly on a point by point basis, since
it permits pipelining and provides better cache performance. Certain
optional quantities such electron pressure, degeneracy parameter, and
thermodynamic derivatives can be calculated by the
\api{physics/Eos/Eos} function if needed. These quantities are
selected for computation based upon a logical mask array provided as
an input argument. A .true. value in the mask array results in the
corresponding quantity being computed and reported back to the calling
function.  Examples of calling the basic implementation \code{Eos} are provided in the API
description, see \api{physics/Eos/Eos}.

The hydrodynamic and burning computations repeatedly call the Eos function to
update pressure and temperature during the course of their
calculation. Typically, values in all the cells of the block need of
be updated in these calls. Since the primary Eos interface requires
the data to be organized as a vector, using it directly could make the
code in the calling unit very cumbersome and error prone. The wrapper
interface, \api{physics/Eos/Eos_wrapped} %\index{equation of state!Eos_wrapped@\code{Eos\_wrapped}}
provides a means by which the
details of translating the data from block to vector and back are
hidden from the calling unit. The wrapper interface permits the caller
to define a section of block by giving the limiting indices along each
dimension. The \code{Eos_wrapped} routine translates the
block section thus described into the vector format of the
\api{physics/Eos/Eos} interface, and upon return translates the
vector format back to the block section. This wrapper routine cannot
calculate the optional derivative quantities.  If they are needed, call
the \code{Eos} routine directly with the optional mask set to true and space allocated
for the returned quantities.



\section{Unit Test}\label{Sec:Eos Unit Test}

The unit test of the Eos function can exercise all three
implementations. Because the Gamma law allows only one species, the setup
required for the three implementations is specific.
To invoke any three-dimensional \unit{Eos}
unit test, the command is:
\begin{quote}
\code{./setup unitTest/Eos/}\metavar{implementation } \code{-auto -3d}
\end{quote}
where \metavar{implementation} is one of \code{Gamma}, \code{Multigamma}, \code{Helmholtz}.
The \unit{Eos} unit test works on the assumption that if the four physical
variables in question (density, pressure, energy and temperature) are
in thermal equilibrium with one another, then applying the equation of state to
any two of them should leave the other two completely
unchanged. Hence, if we initialize density and temperature with some
arbitrary values, and apply the equation of state to them in
\code{MODE\_DENS\_TEMP}, then we should get pressure and energy values that are
thermodynamically consistent with density and temperature. Now after
saving the original temperature value, we apply the equation of state to
density and newly calculated pressure.  The new value of the temperature
should be identical to the saved original value. This verifies that the \unit{Eos}
unit is computing correctly in \code{MODE\_DENS\_PRES} mode. By repeating
this process for the remaining two modes, we can say with great
confidence that the \unit{Eos} unit is functioning normally.

In our implementation of the Eos unit test, the initial conditions applied to the domain
create a gradient for density along the $x$ axis and gradients for
temperature and pressure along the $y$ axis. If the test is being run
for the Multigamma or Helmholtz implementations, then the species are initialized to
have gradients along the $z$ axis.

