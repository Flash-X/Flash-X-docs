
\chapter{VisIt}
\label{Sec:visit}
The developers of \code{Flash-X} also highly recommend VisIt, a free parallel interactive
visualization
package provided by Lawrence Livermore National Laboratory 
(see \url{https://wci.llnl.gov/codes/visit/}).
VisIt runs on Unix and PC platforms, and can handle small desktop-size datasets
as well as very large parallel datasets in the terascale range.  VisIt
provides a native reader to import \code{Flash-X2.5} and \flashx.
Version 1.10 and higher natively
support \flashx.
For VisIt versions 1.8 or less, \flashx support can be obtained by installing
a tarball patch available at 
\url{http://flash.uchicago.edu/site/flashcode/user_support/visit/}.
Full instructions are also available at that site.


\chapter{Serial Flash-X Output Comparison Utility (\code{sfocu})}
\label{Chp:sfocu}

\code{Sfocu} (Serial Flash Output Comparison Utility) is mainly used as part of
an automated testing suite called \code{flashTest} and was introduced in Flash-X
version 2.0 as a replacement for focu.

\code{Sfocu} is a serial utility which examines two Flash-X checkpoint files and
decides whether or not they are ``equal'' to ensure that any changes made to Flash-X do
not adversely affect subsequent simulation output.  By ``equal'', we mean
that

\begin{itemize} \item The leaf-block structure matches -- each leaf block
must have the same position and size in both datasets.

\item The data arrays in the leaf blocks (\code{dens}, \code{pres}...) are
identical.

\item The number of particles are the same, and all floating point particle
attributes are identical.
\end{itemize}

Thus, \code{sfocu} ignores information such as the particular numbering of the
blocks and particles, the timestamp, the build information, and so on.

\code{Sfocu} can read \code{HDF5} and \code{PnetCDF}
Flash-X checkpoint files. Although \code{sfocu} is a
serial program, it is able to do comparisons on the output of large
parallel simulations.  \texttt{Sfocu} has been used on irix, linux,
AIX and OSF1.

\section{Building \code{sfocu}}
\label{Sec:building sfocu}

The process is entirely manual, although Makefiles for certain machines have
been provided. There are a few compile-time options which you set via the
following preprocessor definitions in the Makefile (in the
\code{CDEFINES} macro):

\begin{description}
\item[\code{NO\_HDF5}] build without HDF5 support
\item[\code{NO\_NCDF}] build without PnetCDF support
\item[\code{NEED\_MPI}] certain parallel versions of HDF5 and all versions of PnetCDF
need to be linked with the MPI library.  This adds the necessary
\code{MPI\_Init} and \code{MPI\_Finalize} calls to \code{sfocu}.
There is no advantage to running \code{sfocu} on more than one
processor; it will only give you multiple copies of the same report.
\end{description}


\section{Using \code{sfocu}}
\label{Sec:using sfocu}

The basic and most common usage is to run the command \code{sfocu <file1>
<file2>}. The option \code{-t <dist>} allows a distance tolerance in comparing bounding boxes of blocks 
in two different files to determine which are the same (which have data to compare to one another).
You might need to widen your terminal to view the output,
since it can be over 80 columns. Sample output follows:

\begin{quote}
\begin{tiny}
\begin{samepage}
\begin{verbatim}
A: 2006-04-25/sod_2d_45deg_4lev_ncmpi_chk_0001
B: 2005-12-14/sod_2d_45deg_4lev_ncmpi_chk_0001
Min Error: inf(2|a-b| / max(|a+b|, 1e-99) )
Max Error: sup(2|a-b| / max(|a+b|, 1e-99) )
Abs Error: sup|a-b|
Mag Error: sup|a-b| / max(sup|a|, sup|b|, 1e-99)
Block shapes for both files are: [8,8,1]
Mag-error tolerance: 1e-12
Total leaf blocks compared: 541 (all other blocks are ignored)
-----+------------+-----------++-----------+-----------+-----------++-----------+-----------+-----------+
Var  | Bad Blocks | Min Error ||             Max Error             ||             Abs Error             |
-----+------------+-----------++-----------+-----------+-----------++-----------+-----------+-----------+
     |            |           ||   Error   |     A     |     B     ||   Error   |     A     |     B     |
-----+------------+-----------++-----------+-----------+-----------++-----------+-----------+-----------+
dens | 502        | 0         || 1.098e-11 |  0.424    |  0.424    || 4.661e-12 |  0.424    |  0.424    |
eint | 502        | 0         || 1.1e-11   |  1.78     |  1.78     || 1.956e-11 |  1.78     |  1.78     |
ener | 502        | 0         || 8.847e-12 |  2.21     |  2.21     || 1.956e-11 |  2.21     |  2.21     |
gamc | 0          | 0         || 0         |  0        |  0        || 0         |  0        |  0        |
game | 0          | 0         || 0         |  0        |  0        || 0         |  0        |  0        |
pres | 502        | 0         || 1.838e-14 |  0.302    |  0.302    || 1.221e-14 |  0.982    |  0.982    |
temp | 502        | 0         || 1.1e-11   |  8.56e-09 |  8.56e-09 || 9.41e-20  |  8.56e-09 |  8.56e-09 |
velx | 516        | 0         || 5.985     |  5.62e-17 | -1.13e-16 || 2.887e-14 |  0.657    |  0.657    |
vely | 516        | 0         || 2         |  1e-89    | -4.27e-73 || 1.814e-14 |  0.102    |  0.102    |
velz | 0          | 0         || 0         |  0        |  0        || 0         |  0        |  0        |
mfrc | 0          | 0         || 0         |  0        |  0        || 0         |  0        |  0        |
-----+------------+-----------++-----------+-----------+-----------++-----------+-----------+-----------+
-----+------------+-----------++-----------+-----------+-----------++-----------+-----------+-----------+
Var  | Bad Blocks | Mag Error ||                  A                ||                  B                |
-----+------------+-----------++-----------+-----------+-----------++-----------+-----------+-----------+
     |            |           ||    Sum    |    Max    |    Min    ||    Sum    |    Max    |    Min    |
-----+------------+-----------++-----------+-----------+-----------++-----------+-----------+-----------+
dens | 502        | 4.661e-12 ||  1.36e+04 |  1        |  0.125    ||  1.36e+04 |  1        |  0.125    |
eint | 502        | 6.678e-12 ||  7.3e+04  |  2.93     |  1.61     ||  7.3e+04  |  2.93     |  1.61     |
ener | 502        | 5.858e-12 ||  8.43e+04 |  3.34     |  2        ||  8.43e+04 |  3.34     |  2        |
gamc | 0          | 0         ||  4.85e+04 |  1.4      |  1.4      ||  4.85e+04 |  1.4      |  1.4      |
game | 0          | 0         ||  4.85e+04 |  1.4      |  1.4      ||  4.85e+04 |  1.4      |  1.4      |
pres | 502        | 1.221e-14 ||  1.13e+04 |  1        |  0.1      ||  1.13e+04 |  1        |  0.1      |
temp | 502        | 6.678e-12 ||  0.000351 |  1.41e-08 |  7.75e-09 ||  0.000351 |  1.41e-08 |  7.75e-09 |
velx | 516        | 3.45e-14  ||  1.79e+04 |  0.837    | -6.09e-06 ||  1.79e+04 |  0.837    | -6.09e-06 |
vely | 516        | 2.166e-14 ||  1.79e+04 |  0.838    | -1.96e-06 ||  1.79e+04 |  0.838    | -1.96e-06 |
velz | 0          | 0         ||  0        |  0        |  0        ||  0        |  0        |  0        |
mfrc | 0          | 0         ||  3.46e+04 |  1        |  1        ||  3.46e+04 |  1        |  1        |
-----+------------+-----------++-----------+-----------+-----------++-----------+-----------+-----------+
FAILURE
\end{verbatim}
\end{samepage}
\end{tiny}
\end{quote}

``Bad Blocks'' is the number of leaf blocks where the data was found to differ
between datasets. Four different error measures (min/max/abs/mag) are defined
in the output above. In addition, the last six columns report the sum, maximum
and minimum of the variables in the two files. Note that the sum is physically
meaningless, since it is not volume-weighted. Finally, the last line permits
other programs to parse the \code{sfocu} output easily: when the files
are identical, the line will instead read \code{SUCCESS}.

It is possible for \code{sfocu} to miss machine-precision variations in the data
on certain machines because of compiler or library issues, although this has
only been observed on one platform, where the compiler produced code that
ignored IEEE rules until the right flag was found.

\chapter{Drift}
\label{Chp:drift}
\section{Introduction}
Drift is a debugging tool added to Flash-X to help catch programming mistakes
that occur while refactoring code in a way that \emph{should not} change
numerical behavior.  Historically, simulation checkpoints have been used to
verify that results obtained after a code modification have not affected the
numerics. But if changes are observed, then the best a developer can do to
narrow the bug hunting search space is to look at pairs of checkpoint files
from the two different code bases sequentially.  The first pair to compare
unequal will tell you that somewhere between that checkpoint and its immediate
predecessor something in the code changed the numerics.  Therefor, the search
space can only be narrowed to the limit allow by the checkpointing interval,
which in Flash-X, without clever calls to IO sprinkled about, is at best once per
time cycle.

Drift aims to refine that granularity considerably by allowing comparisons to
be made upon every modification to a block's contents.  To achieve this, drift
intercepts calls to \code{Grid\_releaseBlkPtr}, and inserts into them a step to
checksum each of the variables stored on the block.  Any checksums that do not
match with respect to the last checksums recorded for that block are logged
to a text file along with the source file and line number.  The developer can
then compare two drift logs generated by the different runs using \code{diff}
to find the first log entry that generates unequal checksums, thus telling the
developer which call to \code{Grid\_releaseBlkPtr} first witnessed divergent values.

The following are example excerpts from two drift logs.  Notice the
checksum value has changed for variable \code{dens} on block 18.  This should clue
the developer in that the cause of divergent behavior lies somewhere between
\code{Eos\_wrapped.F90:249} and \code{hy\_ppm\_sweep.F90:533}.

\begin{tabular}{|p{0.4\textwidth}|p{0.4\textwidth}|}
\hline
\begin{scriptsize}
\begin{codeseg}
inst=2036
step=1
src=Eos_wrapped.F90:249
blk=57
 dens E8366F6E49DD1B44
 eint 89D635E5F46E4CE4
 ener C6ED4F02E60C9E8F
 pres 6434628E2D2E24E1
 temp DB675D5AFF7D48B8
 velx 42546C82E30F08B3

inst=2100
step=1
src=hy_ppm_sweep.F90:533
blk=18
 dens A462F49FFC3112DE
 eint 9CD79B2E504C7C7E
 ener 4A3E03520C3536B9
 velx 8193E8C2691A0725
 vely 86C5305CB7DE275E
\end{codeseg}
\end{scriptsize}
&
\begin{scriptsize}
\begin{codeseg}
inst=2036
step=1
src=Eos_wrapped.F90:249
blk=57
 dens E8366F6E49DD1B44
 eint 89D635E5F46E4CE4
 ener C6ED4F02E60C9E8F
 pres 6434628E2D2E24E1
 temp DB675D5AFF7D48B8
 velx 42546C82E30F08B3

inst=2100
step=1
src=hy_ppm_sweep.F90:533
blk=18
 dens 5E52D67C5E93FFF1
 eint 9CD79B2E504C7C7E
 ener 4A3E03520C3536B9
 velx 8193E8C2691A0725
 vely 86C5305CB7DE275E
\end{codeseg}
\end{scriptsize}
\\
\hline
\end{tabular}

\section{Enabling drift}
In Flash-X, drift is disabled by default.  Enabling drift is done by
hand editing the Simulation.h file generated by the setup process.  The directive
line \code{\#define DRIFT_ENABLE 0} should be changed to
\code{\#define DRIFT_ENABLE 1}.  Once this has been changed, a recompilation will be
necessary by executing \code{make}.

With drift enabled, the Flash-X executable will generate log files in the same
directory it is executed in.  These files will be named \code{drift.<rank>.log},
one for each MPI process.

The following runtime parameters are read by drift to control its behavior:

\begin{tabular}{|l|r|p{0.5\textwidth}|}
\hline
Parameter & Default & Description\\
\hline
\code{drift\_trunc\_mantissa} & 2 & The number of least significant mantissa bits to
zero out before hashing a floating point value.  This can be used to stop numerical
noise from altering checksum values.\\
\hline
\code{drift\_verbose\_inst} & 0 & The instance index at which drift should start logging
checksums per call to \code{Grid\_releaseBlkPtr}.  Before this instance is hit, only user
calls to \code{Driver\_driftUnk} will generate log data.  A value of zero means never
log checksums per block.  Instance counting is described below.\\
\hline
\code{drift\_tuples} & .false. & A boolean switch indicating if drift should write
logs in the more human readable "non-tuples" format or the machine friendly "tuples"
format that can be read in by the \code{driftDee} script found in the \code{tools/}
directory.  Generally the "non-tuples" format is a better choice to use with tools such
as \code{diff}.\\
\hline
\end{tabular}

\section{Typical workflow}
Drift has two levels of output verbosity, let us refer to them as verbose and
not verbose.  When in non-verbose mode, drift will only generate output when
directly told to do so through the \code{Driver\_driftUnk} API call.  This call
tells drift to generate a checksum for each \code{unk} variable over all
blocks in the domain and then log those checksums that have changed since the
last call to \code{Driver\_driftUnk}. Verbose mode also generates this
information and additionally includes the per-block checksums for every call
to \code{Grid\_releaseBlkPtr}.  Verbose mode can generate \emph{a lot} of log
data and so should only be activated when the simulation nears the point at
which divergence originates.  This is the reason for the
\code{drift\_verbose\_inst} runtime parameter.

Drift internally maintains an "instance" counter that is incremented with every
intercepted call to \code{Grid\_releaseBlkPtr}.  This is drift's way of enumerating
the program states.  When comparing two drift logs, if the first checksum discrepancy
occurs at instance number 1349 (arbitrary), then it is clear that somewhere between the
1348'th and 1349'th call to \code{Grid\_releaseBlkPtr} a divergent event occurred.

The suggested workflow once drift is enabled is to first run both
simulations with verbose mode off (\code{dirft\_verbse\_inst}=0).  The main
\code{Driver\_evolveFlash} implementations have calls to
\code{Driver\_driftUnk} between all calls to Flash-X unit advancement routines.
So the default behavior of drift will generate multiple unk-wide
checksums for each variable per timestep.  These two drift logs should be
compared to find the first entry with a mismatched checksum.  Each entry
generated by \code{Driver\_driftUnk} will contain an instance range like in the
following:

\begin{codeseg}
step=1
from=Driver_evolveFlash.F90:276
unks inst=1234 to 2345
 dens 9CF3C169A5BB129C
 eint 9573173C3B51CD12
 ener 028A5D0DED1BC399
 ...
\end{codeseg}

The line "\code{unks inst=1349 to 2345}" informs us these checksums were generated
sometime after the 2345'th call to \code{Grid\_releaseBlkPtr}.  Assume this entry
is the first such entry to not match checksums with its counterpart.  Then
we know that somewhere between instance 1234 and 2345 divergence began.  So we
set \code{drift\_verbose\_inst = 1234} in the runtime parameters file of each
simulation and then run them both again.  Now drift will run up to instance
1234 as before, only printing at calls to \code{Driver\_driftUnk}, but starting
with instance 1234 each call to \code{Grid\_releaseBlkPtr} will induce a per
block checksum to be logged as well.  Now these two drift files can be compared
to find the first difference, and hopefully get you on your way to hunting
down the cause of the bug.

\section{Caveats and Annoyances}
The machinery drift uses to intercept calls to \code{Grid\_releaseBlkPtr} is lacking in
sophistication, and as such can put some unwanted constraints on the code base.
The technique used is to declare a preprocessor \code{\#define} in \code{Simulation.h}
to expand occurrences of \code{Grid\_releaseBlkPtr} to something larger that includes
\code{\_\_FILE\_\_} and \code{\_\_LINE\_\_}.  This is how drift is able to
correlate calls to \code{Grid\_releaseBlkPtr} with the originating line of
source code.  Unfortunately this technique places a very specific restriction on the
code once drift is enabled.  The trouble comes from the types of source lines that
may refer to a subroutine without calling it.  The primary offender being \code{use}
statements with \code{only} clauses listing the module members to import into scope.
Because macro expansion is dumb with respect to context, it will expand occurrences
of \code{Grid\_releaseBlkPtr} in these \code{use} statements, turning them into
syntactic rubbish.  The remedy for this issue is to make sure the line
\code{\#include "Simulation.h"} comes after all statements involving
\code{Grid\_releaseBlkPtr} but not calling it, and before all statements that
are calls to \code{Grid\_releaseBlkPtr}.  In practice this is quite easy.  With
only one subroutine per file, there will only be one line like:
\begin{codeseg}
use Grid\_interface, only: ..., Grid\_releaseBlkPtr, ...
\end{codeseg}
and it will come before all calls to \code{Grid\_releaseBlkPtr}, so just move
the \code{\#include "Simulation.h"} after the \code{use} statements.  The following is an example:

\vspace{0.1 in}

\begin{tabular}{|p{0.4\textwidth}|p{0.4\textwidth}|}
\hline
Incorrect & Correct \\
\hline
\begin{scriptsize}
\begin{codeseg}
#include "Simulation.h"
subroutine Flash_subroutine()
  use Grid_interface, only: Grid_releaseBlkPtr
  implicit none
  [...]
  call Grid_releaseBlkPtr(...)
  [...]
end subroutine Flash_subroutine
\end{codeseg}
\end{scriptsize}
&
\begin{scriptsize}
\begin{codeseg}
subroutine Flash_subroutine()
  use Grid_interface, only: Grid_releaseBlkPtr
  implicit none
#include "Simulation.h"
  [...]
  call Grid_releaseBlkPtr(...)
  [...]
end subroutine Flash_subroutine
\end{codeseg}
\end{scriptsize}
\\
\hline
\end{tabular}

\vspace{0.1 in}

If such a solution is not possible because no separation between all \code{use}
and \code{call} statements exists, then there are two remaining courses of action
to get the source file to compile.  One, hide these calls to \code{Grid\_releaseBlkPtr}
from drift by forceably disabling the macro expansion.  To do so, just add the
line \code{\#undef Grid\_releaseBlkPtr} after \code{\#include "Simulation.h"}.  The
second option is to carry out the macro expansion by hand.  This also requires
disabling the macro with the undef just mentioned, but then also rewriting each
call to \code{Grid\_releaseBlkPtr} just as the preprocessor would.  Please consult
\code{Simulation.h} to see the text that gets substituted in for \code{Grid\_releaseBlkPtr}.
