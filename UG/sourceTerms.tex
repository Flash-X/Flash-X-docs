\chapter{Local Source Terms}
\label{Sec:source terms}

The \code{physics/sourceTerms} organizational directory contains several units 
that implement forcing terms.  The \code{Burn}, \code{Stir}, \code{Ionize}, and
\code{Diffuse} units contain implementations in \flashx.  Two other units,
\code{Cool} and \code{Heat}, contain only stub level routines in their API.

\section{Burn Unit}
\label{Sec:burn}

The nuclear burning implementation of the \code{Burn} unit uses a
sparse-matrix semi-implicit ordinary differential equation (ODE)
solver to calculate the nuclear burning rate and to update the fluid
variables accordingly (Timmes 1999). The primary interface routines
for this unit are \api{physics/sourceTerms/Burn/Burn_init}, which
sets up the nuclear isotope tables needed by the
unit, and \api{physics/sourceTerms/Burn/Burn}, which
calls the ODE solver and updates the hydrodynamical variables in a
single row of a single block.
The routine \api{physics/sourceTerms/Burn/Burn_computeDt} may limit the computational timestep because of burning considerations.
There is also a helper routine \code{Simulation/\-SimulationComposition/\-Simulation_initSpecies}
(see \api{Simulation/Simulation_initSpecies})
which provides the properties of ions included in the burning network.


\begin{comment}
% \begin{table}
% \caption{ \label{Tab:burn parameters} Runtime parameters used with the
% \code{Burn} unit.}
% \begin{center}
% \begin{tabular}{lllp{3.5in}}
% Variable    & Type      & Default   & Description\\
% \hline
% \code{algebra}         & integer   & 1 & Choice of linear algebra package for sparse matrices \\
% \code{enucDtFactor}    & real      & $1.0\times10^30$ & Time step limiter based on energy generation from burning \\
% \code{nuclearTempMin}  & real      & $1.1\times10^8$ & Minimum temperature in K    for burning to be allowed\\
% \code{nuclearTempMax}  & real      & $1.0\times10^{12}$
%                           & Maximum temperature in K   for burning to be allowed\\
% \code{nuclearDensMin}  & real      & $1.0\times10^{-10}$
%                           & Minimum    density (g~cm$^{-3}$) for burning to be allowed\\
% \code{nuclearDensMax}  & real      & $1.0\times10^{14}$
%                           & Maximum  density (g~cm$^{-3}$) for  burning to be allowed\\
% \code{nuclearNI56Max}  & real      & 0.4         & Maximum Ni$^{56}$ mass
%                             fraction for burning                    to be allowed\\
% \code{odeStepper}      & integer   & 1 & Choice of implicit integration scheme for evolving reaction network\\
% \code{useBurnTable}    & boolean   & false & Evaluate reaction rates directly, or by using a table approximation \\
% \code{useShockBurning} & boolean   & ?? & ?? \\
% \hline
% \end{tabular}
% \end{center}
% \end{table}
\end{comment}

\subsection{Algorithms}

~Modeling thermonuclear flashes typically requires the energy
generation rate due to nuclear burning over a large range of
temperatures, densities and compositions. The average energy generated
or lost over a period of time is found by integrating a system of
ordinary differential equations (the nuclear reaction network) for the
abundances of important nuclei and the total energy release.  In some
contexts, such as supernova models, the abundances themselves
are also of interest. In either case, the coefficients that appear in
the equations are typically extremely sensitive to temperature.  The
resulting stiffness of the system of equations requires the use of an
implicit time integration scheme.

A user can choose between two implicit integration methods and two
linear algebra packages in Flash-X. The runtime parameter
\rpi{Burn/odeStepper}
controls which integration method is used in the simulation. The
choice \code{odeStepper = 1} is the default and invokes a
Bader-Deuflhard scheme.  The choice \code{odeStepper = 2} invokes a
Kaps-Rentrop or Rosenbrock scheme.
The runtime parameter \rpi{Burn/algebra} controls which
linear algebra package is used in the simulation.  The choice
\code{algebra = 1} is the default and invokes the sparse matrix MA28 package.
The choice \code{algebra = 2} invokes the GIFT linear algebra routines.
While any combination of the integration methods and linear algebra
packages will produce correct answers, some combinations may execute
more efficiently than others for certain types of
simulations.  No general rules have been found for best combination
for a given simulation. The most efficient combination depends on the
timestep being taken, the spatial resolution of the model, the values
of the local thermodynamic variables, and the composition. Users are
advised to experiment with the various combinations to determine the
best one for their simulation.  However, an extensive analysis was performed in
the Timmes paper cited below.

Timmes (1999) reviewed several methods for solving stiff nuclear
reaction networks, providing the basis for the reaction network solvers
included with Flash-X.  The scaling properties and behavior of three
semi-implicit time integration algorithms (a traditional first-order
accurate Euler method, a fourth-order accurate Kaps-Rentrop / Rosenbrock method,
and a variable order Bader-Deuflhard method) and eight linear algebra
packages (LAPACK, LUDCMP, LEQS, GIFT, MA28, UMFPACK, and Y12M) were
investigated by running each of these 24 combinations on seven
different nuclear reaction networks (hard-wired 13- and 19-isotope
networks and soft-wired networks of 47, 76, 127, 200, and 489
isotopes).  Timmes' analysis suggested that the best balance of
accuracy, overall efficiency, memory footprint, and ease-of-use was
provided by the two integration methods (Bader-Deuflhard and Kaps-Rentrop)
and the two linear algebra packages (MA28 and GIFT) that are provided
with the Flash-X code.


\subsection{Reaction networks}


We begin by describing the equations solved by the nuclear burning
unit. We consider material that may be described by a density
$\rho$ and a single temperature $T$ and contains a number of isotopes
$i$, each of which has $Z_{i}$ protons and $A_i$ nucleons (protons +
neutrons). Let $n_i$ and $\rho_i$ denote the number and mass density,
respectively, of the $i$th isotope, and let $X_i$ denote its mass
fraction, so that
\begin{equation}
X_i = \rho_i/\rho = n_i A_i/(\rho N_A)\ ,
\end{equation}
where $N_A$ is Avogadro's number.
Let the molar abundance of
the $i$th isotope be
\begin{equation}
Y_i = X_i/A_i = n_i/(\rho N_A)\ .
\end{equation}
Mass conservation is then expressed by
\begin{equation}
\label{Eqn:mass conservation}
\sum_{i=1}^N X_i = 1~.
\end{equation}
At the end of each timestep, Flash-X checks that the stored abundances
satisfy \eqref{Eqn:mass conservation} to machine precision in order
to avoid the unphysical buildup (or decay) of the abundances or
energy generation rate. Roundoff errors in this equation can lead to
significant problems in some contexts (\eg, classical nova
envelopes), where trace abundances are important.

The general continuity equation for the $i$th isotope is given in
Lagrangian formulation by
\begin{equation}
\label{Eqn:isotope continuity}
\drvf {Y_i} {t} + \nabla \cdot \left ( Y_i \avec {V}_i \right ) = \dt {R_i}\ .
\end{equation}
In this equation
$\dt {R_i}$ is the total reaction rate due to
all binary reactions of the form {\it i(j,k)l},
\begin{equation}
\label{Eqn:binary rate}
\dt {R_{i}}
 = \sum_{j,k}
              Y_{l} Y_{k} \lambda _{kj}(l)  - Y_{i} Y_{j} \lambda _{jk}(i)\ ,
\end{equation}
where $\lambda _{kj}$ and $\lambda _{jk}$ are the reverse (creation)
and forward (destruction) nuclear reaction rates, respectively.
Contributions from three-body reactions, such as the triple-$\alpha$
reaction, are easy to append to \eqref{Eqn:binary rate}.  The mass
diffusion velocities $\avec{V}_i$ in \eqref{Eqn:isotope continuity}
are obtained from the solution of a multicomponent diffusion
equation (Chapman \& Cowling 1970; Burgers 1969; Williams 1988) and
reflect the fact that mass diffusion processes arise from pressure,
temperature, and/or abundance gradients as well as from external
gravitational or electrical forces.

The case $\avec{V}_i\equiv 0$ is important for two reasons. First,
mass diffusion is often unimportant when compared to other transport
processes, such as thermal or viscous diffusion (\ie, large Lewis
numbers and/or small Prandtl numbers). Such a situation obtains, for
example, in the study of laminar flame fronts propagating through
the quiescent interior of a white dwarf. Second, this case permits
the decoupling of the reaction network solver from the
hydrodynamical solver through the use of operator splitting, greatly
simplifying the algorithm.  This is the method used by the default
Flash-X distribution. Setting $\avec{V}_i\equiv 0$ transforms
\eqref{Eqn:isotope continuity} into
\begin{equation}
\label{Eqn:nucrate 1}
\drvf {Y_i} {t} = \dt {R_i}\ ,
\end{equation}
which may be written in the more compact, standard form
\begin{equation}
\label{Eqn:nucrate 2}
\dot {{\bf y}} = {\bf f} \ ({\bf y})\ .
\end{equation}
Stated another way, in the absence of mass diffusion or advection,
any changes to the fluid composition are due to local processes.

Because of the highly nonlinear temperature dependence of the
nuclear reaction rates and because the abundances themselves often
range over several orders of magnitude in value, the values of the
coefficients which appear in \eqref{Eqn:nucrate 1} and
\eqref{Eqn:nucrate 2} can vary quite significantly.  As a result,
the nuclear reaction network equations are ``stiff.'' A system of
equations is stiff when the ratio of the maximum to the minimum
eigenvalue of the Jacobian matrix ${\tilde {\bf
J}}\equiv\partial{{\bf f}}/\partial{{\bf y}}$ is large and
imaginary. This means that at least one of the isotopic abundances
changes on a much shorter timescale than another.  Implicit or
semi-implicit time integration methods are generally necessary to
avoid following this short-timescale behavior, requiring the
calculation of the Jacobian matrix.

It is instructive at this point to look at an example of how
\eqref{Eqn:nucrate 1} and the associated Jacobian matrix are
formed. Consider the $^{12}$C($\alpha$,$\gamma$)$^{16}$O reaction,
which competes with the triple-$\alpha$ reaction during helium
burning in stars. The rate $R$ at which this reaction proceeds is
critical for evolutionary models of massive stars, since it
determines how much of the core is carbon and how much of the core
is oxygen after the initial helium fuel is exhausted.  This reaction
sequence contributes to the right-hand side of \eqref{Eqn:nucrate
2} through the terms
\begin{eqnarray}
\nonumber
\dot {Y} (^4He)   & =& - Y(^4He) \ Y(^{12}C) \ R + \ldots \\
\dot {Y} (^{12}C) & =& - Y(^4He) \ Y(^{12}C) \ R \ + \ldots  \\
\nonumber
\dot {Y} (^{16}O) & =& + Y(^4He) \ Y(^{12}C) \ R \ + \ldots ,
\end{eqnarray}
where the ellipses indicate additional terms coming from other
reaction sequences.  The minus signs indicate that helium and carbon
are being destroyed, while the plus sign indicates that oxygen is
being created. Each of these three expressions contributes two terms
to the Jacobian matrix
${\tilde {\bf J}}$=$\partial{{\bf f}}/\partial{{\bf y}}$
\begin{eqnarray}
\nonumber
J(^4He,^4He)     = - Y(^{12}C) \ R \ + \ldots  \hskip 0.5in
&
J(^4He,^{12}C)   = - Y(^4He) \ R \ + \ldots \\
J(^{12}C,^4He)   = - Y(^{12}C) \ R \ + \ldots \hskip 0.5in
&
J(^{12}C,^{12}C) = - Y(^4He) \ R \ + \ldots   \\
\nonumber
J(^{16}O,^4He)   = + Y(^{12}C) \ R \ + \ldots \hskip 0.5in
&
J(^{16}O,^{12}C) = + Y(^4He) \ R \ + \ldots .
\end{eqnarray}
Entries in the Jacobian matrix represent the flow, in number of nuclei
per second, into (positive) or out of (negative) an isotope.  All of the
temperature and density dependence is included in the reaction rate
$R$.  The Jacobian matrices that arise from nuclear reaction networks
are neither positive-definite nor symmetric, since the forward and
reverse reaction rates are generally not equal. In addition, the
magnitudes of the matrix entries change as the abundances,
temperature, or density change with time.

This release of \flashx contains three reaction networks.
A seven-isotope alpha-chain (\code{Iso7}) is useful for problems that
do not have enough memory to carry a larger set of isotopes.
The 13-isotope $\alpha$-chain plus heavy-ion reaction network (\code{Aprox13})
is suitable for most multi-dimensional simulations of stellar phenomena,
where having a reasonably accurate energy generation rate is of primary
concern.  The 19-isotope reaction network (\code{Aprox19}) has the same
$\alpha$-chain and heavy-ion reactions as the 13-isotope network, but it
includes additional isotopes to accommodate some types of hydrogen burning
(PP chains and steady-state CNO cycles), along with some aspects of
photo-disintegration into $^{54}$Fe. This 19 isotope reaction network
is described in Weaver, Zimmerman, \& Woosley (1978).

%The \code{Ppcno}
%network includes reactions for the pp and CNO cycles.  A number of simple
%single-reaction networks are also provided.

The networks supplied with Flash-X are examples of a ``hard-wired'' reaction
network, where each of the reaction sequences are carefully entered
by hand.  This approach is suitable for small networks, when minimizing
the CPU time required to run the reaction network is a primary
concern, although it suffers the disadvantage of inflexibility.

The MA28 sparse matrix package used by Flash-X is described by Duff,
Erisman, \& Reid (1986) and is used as an external library.  This package, which has been described
as the ``Coke classic'' of sparse linear algebra packages, uses a direct
-- as opposed to an iterative -- method for solving linear systems.  Direct
methods typically divide the solution of $\tilde{{\bf A}} \cdot {\bf
x} = {\bf b}$ into a symbolic LU decomposition, a numerical LU
decomposition, and a backsubstitution phase.  In the symbolic LU
decomposition phase, the pivot order of a matrix is determined, and a
sequence of decomposition operations that minimizes the amount of
fill-in is recorded. Fill-in refers to zero matrix elements which
become nonzero (\eg, a sparse matrix times a sparse matrix is
generally a denser matrix).  The matrix is not decomposed; only the
steps to do so are stored. Since the nonzero pattern of a chosen
nuclear reaction network does not change, the symbolic LU
decomposition is a one-time initialization cost for reaction networks.
In the numerical LU decomposition phase, a matrix with the same pivot
order and nonzero pattern as a previously factorized matrix is
numerically decomposed into its lower-upper form.  This phase must be
done only once for each set of linear equations.  In the
backsubstitution phase, a set of linear equations is solved with the
factors calculated from a previous numerical decomposition.  The
backsubstitution phase may be performed with as many right-hand sides
as needed, and not all of the right-hand sides need to be known in
advance.

MA28 uses a combination of nested dissection and frontal envelope
decomposition to minimize fill-in during the factorization stage.  An
approximate degree update algorithm that is much faster
(asymptotically and in practice) than computing the exact degrees is
employed.  One continuous real parameter sets the amount of searching
done to locate the pivot element. When this parameter is set to zero,
no searching is done and the diagonal element is the pivot, while when
set to unity, partial pivoting is done.  Since the matrices
generated by reaction networks are usually diagonally dominant, the
routine is set in Flash-X to use the diagonal as the pivot
element. Several test cases showed that using partial pivoting did not
make a significant accuracy difference but was less efficient, since
a search for an appropriate pivot element had to be performed.  MA28
accepts the nonzero entries of the matrix in the \hbox{$(i, j,
a_{i,j}$)} coordinate system and typically uses 70$-$90\% less
storage than storing the full dense matrix.


\subsubsection{Two time integration methods}


One of the time integration methods used by Flash-X for evolving the
reaction networks is a 4th-order accurate Kaps-Rentrop, or Rosenbrock method. In
essence, this method is an implicit Runge-Kutta algorithm.  The
reaction network is advanced over a timestep $h$ according to
\begin{equation}
\label{Eqn:kr1}
{\bf y}^{n+1} = {\bf y}^n + \sum_{i=1}^4 b_i \Delta_i
\ ,
\end{equation}
where the four vectors $\Delta^i$ are found from successively
solving the four matrix equations

\begin{eqnarray}
(\tilde{{\bf 1}}/\gamma h - \tilde{{\bf J}}) \cdot \Delta_1 & = &
{\bf f} ({\bf y}^n)\\
(\tilde{{\bf 1}}/\gamma h - \tilde{{\bf J}}) \cdot \Delta_2 & = &
{\bf f} ({\bf y}^n + a_{21}\Delta_1) + c_{21}\Delta_1/h\\
(\tilde{{\bf 1}}/\gamma h - \tilde{{\bf J}}) \cdot \Delta_3 & = &
{\bf f} ({\bf y}^n + a_{31}\Delta_1 + a_{32}\Delta_2) +
(c_{31}\Delta_1 + c_{32}\Delta_2)/h\\
(\tilde{{\bf 1}}^\gamma h - \tilde{{\bf J}}) \cdot \Delta_4 & = &
{\bf f} ({\bf y}^n + a_{31}\Delta_1 + a_{32}\Delta_2) +
(c_{41}\Delta_1 + c_{42}\Delta_2 + c_{43}\Delta_3)/h
\ .
\end{eqnarray}
$b_i$, $\gamma$, $a_{ij}$, and $c_{ij}$ are
fixed constants of the method.  An estimate of the accuracy of the
integration step is made by comparing a third-order solution with a
fourth-order solution, which is a significant improvement over the
basic Euler method.  The minimum cost of this method $-$ which applies
for a single timestep that meets or exceeds a specified integration
accuracy $-$ is one Jacobian evaluation, three evaluations of the
right-hand side, one matrix decomposition, and four backsubstitutions.
Note that the four matrix equations represent a staged set of linear equations
($\Delta_4$ depends on $\Delta_3 \ldots$ depends on $\Delta_1$).  Not
all of the right-hand sides are known in advance. This general feature
of higher-order integration methods impacts the optimal choice of a
linear algebra package.  The fourth-order Kaps-Rentrop routine in
Flash-X makes use of the routine GRK4T given by Kaps \& Rentrop
(1979).

Another time integration method used by Flash-X for evolving the reaction
networks is the variable order Bader-Deuflhard method (\eg, Bader \&
Deuflhard 1983).  The reaction network is advanced
over a large timestep $H$ from ${\bf y}^n$ to ${\bf y}^{n+1}$ by the
following sequence of matrix equations. First,
\begin{eqnarray}
\nonumber
h & = & H/m \\
\label{Eqn:BD 1}
(\tilde{{\bf 1}} - \tilde{{\bf J}}) \cdot \Delta_0 & = & h {\bf f}
({\bf y}^n) \\
\nonumber
{\bf y}_1 & = &{\bf y}^n + \Delta_0\ .
\end{eqnarray}
Then from $k=1,2,\ldots,m-1$
\begin{eqnarray}
\nonumber
(\tilde{{\bf 1}} - \tilde{{\bf J}}) \cdot {\bf x} & = &
h {\bf f}({\bf y}_{k}) - \Delta_{k-1}  \\
\Delta_k & = & \Delta_{k-1} + 2 {\bf x} \\
\nonumber
{\bf y}_{k+1} & = &{\bf y}_k + \Delta_k \ ,
\end{eqnarray}
and closure is obtained by the last stage
\begin{eqnarray}
\nonumber
(\tilde{{\bf 1}} - \tilde{{\bf J}}) \cdot \Delta_m & = &
h [ {\bf f} ({\bf y}_m)  - \Delta_{m-1} ] \\
\label{Eqn:BD 3}
{\bf y}^{n+1} & = &{\bf y}_m + \Delta_m \ .
\end{eqnarray}
This staged sequence of matrix equations is executed at least twice
with $m=2$ and $m=6$, yielding a fifth-order method.  The sequence
may be executed a maximum of seven times, which yields a
fifteenth-order method.  The exact number of times the staged
sequence is executed depends on the accuracy requirements (set to
one part in 10$^6$ in Flash-X) and the smoothness of the solution.
Estimates of the accuracy of an integration step are made by
comparing the solutions derived from different orders.  The minimum
cost of this method --- which applies for a single timestep that met
or exceeded the specified integration accuracy --- is one Jacobian
evaluation, eight evaluations of the right-hand side, two matrix
decompositions, and ten backsubstitutions.  This minimum cost can be
increased at a rate of one decomposition (the expensive part) and
$m$ backsubstitutions (the inexpensive part) for every increase in
the order $2k+1$. The cost of increasing the order is compensated
for, hopefully, by being able to take correspondingly larger (but
accurate) timestep.  The controls for order versus step size are a
built-in part of the Bader-Deuflhard method.  The cost per step of
this integration method is at least twice as large as the cost per
step of either a traditional first-order accurate Euler method or
the fourth-order accurate Kaps-Rentrop discussed above. However, if
the Bader-Deuflhard method can take accurate timesteps that are at
least twice as large, then this method will be more efficient
globally. Timmes (1999) shows that this is typically (but not
always!) the case. Note that in \eqnsref{Eqn:BD 1}{Eqn:BD 3}, not
all of the right-hand sides are known in advance, since the sequence
of linear equations is staged. This staging feature of the
integration method may make some matrix packages, such as MA28, a
more efficient choice.


The Flash-X runtime parameter \rpi{Burn/odeStepper}
controls which integration method is used in the simulation. The choice
\code{odeStepper = 1} is the default and invokes the variable order
Bader-Deuflhard scheme.  The choice \code{odeStepper = 2} invokes the
fourth order Kaps-Rentrop / Rosenbrock scheme.

\subsection{Detecting shocks}

For most astrophysical detonations, the shock structure is so thin
that there is insufficient time for burning to take place within the
shock.  However, since numerical shock structures tend to be much
wider than their physical counterparts, it is possible for a significant
amount of burning to occur within the shock.  Allowing this to happen
can lead to unphysical results.  The burner unit includes a
multidimensional shock detection algorithm that can be used to prevent
burning in shocks.  If the \rpi{Burn/useShockBurn} parameter is set to
\code{.false.}, this algorithm is used to detect shocks in the
Burn unit and to switch off the burning in shocked
cells.

Currently, the shock detection algorithm supports Cartesian and 2-dimensional
cylindrical coordinates.  The basic algorithm is to compare the jump
in pressure in the direction of compression (determined by looking at
the velocity field) with a shock parameter (typically 1/3).  If the
total velocity divergence is negative and the relative pressure jump
across the compression front is larger than the shock parameter, then
a cell is considered to be within a shock.

This computation is done on a block by block basis.  It is important
that the velocity and pressure variables have up-to-date guard cells,
so a guard cell call is done for the burners only if we are detecting
shocks ({\it i.e.} \code{useShockBurning = .false.}).


\subsection{Energy generation rates and reaction rates}

The instantaneous energy generation rate is given by the sum
\begin{equation}
\dot {\epsilon}_{\rm nuc} = N_A \ \sum_i \ \drvf {Y_{i}} {t} \ .
\end{equation}
Note that a nuclear reaction network does not need to be evolved in
order to obtain the instantaneous energy generation rate, since only
the right hand sides of the ordinary differential equations need to be
evaluated. It is more appropriate in the Flash-X program to use the
average nuclear energy generated over a timestep
\begin{equation}
\dot {\epsilon}_{\rm nuc} = N_A \ \sum_i \ {\Delta Y_i \over \Delta t}\ .
\end{equation}
In this case, the nuclear reaction network does need to be evolved.
The energy generation rate, after subtraction of any neutrino losses,
is returned to the Flash-X program for use with the operator splitting
technique.

The tabulation of Caughlan \& Fowler (1988) is used in Flash-X for most
of the key nuclear reaction rates.  Modern values for some of the
reaction rates were taken from the reaction rate library of Hoffman
(2001, priv.\ comm.).  A user can choose between two reaction rate
evaluations in Flash-X.  The runtime parameter \rpi{Burn/useBurnTable} controls
which reaction rate evaluation method is used in the simulation. The
choice \code{useBurnTable = 0} is the default and evaluates the
reaction rates from analytical expressions.  The choice \code{useBurnTable =
1} evaluates the reactions rates from table interpolation. The
reaction rate tables are formed on-the-fly from the analytical
expressions.  Tests on one-dimensional detonations and hydrostatic
burnings suggest that there are no major differences in the abundance
levels if tables are used instead of the analytic expressions; we find
less than 1\% differences at the end of long timescale runs.  Table
interpolation is about 10 times faster than evaluating the analytic
expressions, but the speedup to Flash-X is more modest, a few percent at
best, since reaction rate evaluation never dominates in a real
production run.

Finally, nuclear reaction rate screening effects as formulated by
Wallace {\it et al.} (1982) and decreases in the energy generation rate
$\dot {\epsilon}_{\rm nuc}$ due to neutrino losses as given by Itoh {\it et
al.} (1996) are included in Flash-X.

\subsection{Temperature-based timestep limiting}

When using explicit hydrodynamics methods, a timestep
limiter must be used to ensure the stability of the numerical solution.
The standard CFL limiter is always used when an explicit hydrodynamics
unit is included in Flash-X.  This constraint does not allow any
information to travel more than one computational cell per timestep.
When coupling burning with the hydrodynamics, the CFL timestep may be so
large compared to the burning timescales that the nuclear energy
release in a cell may exceed the existing internal energy in that
cell.  When this happens, the two operations (hydrodynamics and
nuclear burning) become decoupled.

% This enucDtFactor is supposed be less than 1 via the documentation.  But the CONFIG implies the
%  default is 1.0E+30.  Too much discrepancy to figure out now.
To limit the timestep when burning is performed, an additional constraint is imposed.   The limiter tries to force the energy generation from burning to be smaller than the internal energy in a cell.  The runtime parameter \rpi{Burn/enucDtFactor} controls this ratio.  The timestep limiter is calculated as
\begin{equation}
\Delta t_{burn} = \code{enucDtFactor} \cdot \frac{E_{int}}{E_{nuc}}
\end{equation}
where $E_{nuc}$ is the nuclear energy, expressed as energy per volume per time, and
$E_{int}$ is the internal energy per volume.
For good coupling between the hydrodynamics and burning,
\code{enucDtFactor} should be $< 1$. The default value is kept
artificially high so that in most simulations the time limiting due to
burning is turned off. Care must be exercised in the use of this routine.


\begin{comment} %this stuff all removed in flash3
To help fix this problem, the
timestep may be determined by the change in temperature in a cell.  Flash-X
includes a temperature based timestep limiter that tries to constrain
the change in temperature in a cell to be less than a user defined
parameter.  To use this limiter, set \code{itemp\_limit = 1} and
specify the fractional temperature change \rpi{Driver/temp_factor} you are
willing to tolerate.
\marginpar{Why is this RT parameter in Driver?}
While there is no guarantee that the temperature change
will be smaller than this, since the timestep was already taken by the time
this was computed, this method is successful in restoring coupling between
the hydrodynamics and burning operators.  This timestep will be computed as
\begin{equation}
\Delta t = \code{temp\_factor} \cdot \frac{T}{\Delta T} \cdot
{\Delta t_{old}}~,
 \end{equation}
where $\Delta t$ is the difference in the temperature of a cell from one
timestep to the next, and $\Delta t_{old}$ is the last timestep.  To prevent
the timestep from varying wildly from one step to the next, it is
useful to force the maximum change in timestep to be a small factor
over the previous one through the \rpi{Driver/tstep_change_factor}
parameter.
\end{comment}



