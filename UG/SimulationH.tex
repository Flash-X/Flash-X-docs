\chapter{The \code{Simulation.h} file}
\label{Chp:Simulation.h}

\code{Simulation.h}%\index{Simulation.h}
is a critical header file in \flashx which
holds many of the key quantities related to the particular
simulation. The \code{Simulation.h} file is written by the setup script
and should not be modified by the user.  
The \code{Simulation.h} file will be different
for various applications.  When the setup script is building an
application, it parses the \code{Config} files, collects definitions of
variables, fluxes, grid vars, species, and mass scalars, and writes 
a symbol (an index into one of the data structures maintained by
the \unit{Grid} unit)
for each defined entity to the \code{Simulation.h} header file.
This chapter explains these symbols and some of the other 
important quantities and indices defined in the \code{Simulation.h} file.

%%\section{Flash Pre-processor symbols}

%begin{latexonly}
\newcommand{\htabopt}[1]{\multicolumn{2}{l}{\code{#1}}\\*[1ex]}
\newcommand{\htr}{\\[1ex]}
%end{latexonly}
% \begin{htmlonly}
% \newcommand{\htabopt}[1]{\multicolumn{2}{l}{\code{#1}}\\}
% \newcommand{\htr}{\\}
% \end{htmlonly}


\section{\code{UNK}, \code{FACE(XYZ)} Dimensions}
\label{Sec:FlashHdimensions}
Variables in a simulation are stored and manipulated by the
\unit{Grid} unit.  The basic data structures in the \unit{Grid} unit are
5-dimensional arrays: \code{unk}, \code{facex}, \code{facey}, and
\code{facez}.  The
array \code{unk} stores the cell-centered values of various quantities
(density, pressure, etc.).  The \code{facex}, \code{facey} and
\code{facez} variables store face-centered values of some or all of
the same quantities.  Face centered variables are commonly used in Magnetohydrodynamics (MHD)
simulations to hold vector-quantity fields.  The first dimension of
each of these arrays indicates the variable, the 2nd, 3rd, and 4th
dimensions indicate the cell location in a block, and the 5th dimension is
the block identifier for the local processor.  The size of the block,
dimensions of the domain, and other parameters which influence the
\unit{Grid} data structures are defined in \code{Simulation.h}.

\begin{longtable}{p{0.04\textwidth}p{0.9\textwidth}}
 \htabopt{NXB,NYB,NZB} & The number of interior cells in the x,y,z-dimension
per block \htr
 \htabopt{MAXCELLS} & The maximum of (NXB,NYB,NZB)\htr

 \htabopt{MAXBLOCKS} & The maximum number of blocks
which can be allocated in a single processor \htr
 \htabopt{GRID\_(IJK)LO} & The index of the lowest numbered cell in the
x,y,z-direction in a block (not including guard cells) \htr
 \htabopt{GRID\_(IJK)HI} & The index of the highest numbered cell in the
x,y,z-direction in a block (not including guard cells) \htr
 \htabopt{GRID\_(IJK)LO\_GC} & The index of the lowest numbered cell in the
x,y,z-direction in a block (including guard cells) \htr
 \htabopt{GRID\_(IJK)HI\_GC} & The index of the highest numbered cell in the
x,y,z-direction in a block (including guard cells) \htr
 \htabopt{NGUARD} & The number of guard cells in each dimension.

\end{longtable}
All of these constants have meaning when operating in FIXEDBLOCKSIZE
mode%\index{FIXEDBLOCKSIZE mode} only.
FIXEDBLOCKSIZE mode is when the sizes and the 
block bounds are determined at compile time.  In NONFIXEDBLOCKSIZE 
mode%\index{NONFIXEDBLOCKSIZE mode},
the block sizes and the block
bounds are determined at runtime.  \Paramesh always runs in
FIXEDBLOCKSIZE mode, while the Uniform Grid can be run in either FIXEDBLOCKSIZE or
NONFIXEDBLOCKSIZE mode.  See \secref{Sec:ListSetupArgs} and 
\secref{Sec:NONFIXEDBLOCKSIZE} for more information.

\section{Property Variables, Species and Mass Scalars}
\label{Sec:FlashHvariables}
The \code{unk} data structure stores, in order, property variables (like density,
pressure, temperature), the mass fraction of species, and 
mass scalars \footnote{See \eqref{eq:massscalar} for more information about mass scalars}.  
However, in \flashx the user does not need to be intimately aware
of the \code{unk} array layout, as starting
and ending indices of these groups of quantities are defined in \code{Simulation.h}.  
The following
pre-processor symbols define the indices of the various quantities
related to a given cell. These symbols are primarily used to perform
some computation with all property variables, species mass fractions,
or all mass scalars.

\begin{longtable}{p{0.04\textwidth}p{0.9\textwidth}}
 \htabopt{NPROP\_VARS} & The number of property variables in the simulation
 \htr
 \htabopt{NSPECIES} & The total number of species in the simulation \htr
 \htabopt{NMASS\_SCALARS} & The number of mass scalars in the simulation \htr
 \htabopt{NUNK\_VARS} & The total number of quantities stored for
 each cell in the simulation. This equals \code{NPROP\_VARS + NSPECIES + NMASS\_SCALARS} \htr
 \htabopt{PROP\_VARS\_BEGIN,PROP\_VARS\_END} &
 The indices in the \code{unk} array used for property variable data \htr
 \htabopt{SPECIES\_BEGIN,SPECIES\_END} &
 The indices in the \code{unk} array used for species data \htr
 \htabopt{MASS\_SCALARS\_BEGIN,MASS\_SCALARS\_END} &
 The indices in the \code{unk} array used for mass scalars data \htr
 \htabopt{UNK\_VARS\_BEGIN,UNK\_VARS\_END} &
 The low and high indices for the \code{unk} array \htr
\end{longtable}

The indices where specific properties (\eg, density) are stored can
also be accessed via pre-processor symbols.  All properties are
declared in \code{Config} files and consist of 4 letters. For example, 
if a \code{Config} file declares a ``dens'' variable, its index in
the \code{unk} array is available via the pre-processor symbol
\code{DENS\_VAR} (append \code{\_VAR} to the uppercase name of the variable)
which is guaranteed to be an integer. The same is true for species and mass
scalars. In the case of species, the pre-processor symbol is created by
appending \code{\_SPEC} to the uppercase name of the species (\eg,
\code{SF6\_SPEC}, \code{AIR\_SPEC}). Finally, for mass scalars,
\code{\_MSCALAR} is appended to the uppercase name of the mass
scalars.  



\section{Fluxes}
\label{Sec:FlashHfluxes}
The fluxes are stored in their own data structure and are only necessary
when an adaptive grid is in use.  The index order works in much the
same way as with the \code{unk} data structure.  There are the traditional
property fluxes, like density, pressure, \etc  Additionally, there are 
species fluxes and mass scalars fluxes. The name of the pre-processor symbol is
assembled by 
appending \code{\_FLUX} to the uppercase name of the declared flux (\eg,
\code{EINT\_FLUX}, \code{U\_FLUX}). 
For flux species and flux mass scalars,
the suffix \code{\_FLUX\_SPECIES} and \code{\_FLUX\_MSCALAR} are appended to 
the uppercase names of flux species and flux mass scalars, respectively, as declared in the
\code{Config} file.
Useful defined variables are calculated as follows:


\begin{longtable}{p{0.04\textwidth}p{0.9\textwidth}}
 \htabopt{NPROP\_FLUX} & The number of property variables in the simulation
 \htr
 \htabopt{NSPECIES\_FLUX} & The total number of species in the simulation \htr
 \htabopt{NMASS\_SCALARS\_FLUX} & The number of mass scalars in the simulation \htr
 \htabopt{NFLUXES} & The total number of quantities stored for
 each cell in the simulation. This equals \code{(NPROP\_FLUX + NSPECIES\_FLUX + NMASS\_SCALARS\_FLUX)} \htr
%%% This previously said  NFLUX_VARS   and       NPROP\_FLUXES + NFLUX\_SPECIES + NFLUX\_MASS\_SCALARS
 \htabopt{PROP\_FLUX\_BEGIN,PROP\_FLUX\_END} &
 The indices in the \code{fluxes} data structure used for property variable data \htr
 \htabopt{SPECIES\_FLUX\_BEGIN,SPECIES\_FLUX\_END} &
 The indices in the \code{fluxes} data structure used for species data \htr
 \htabopt{MASS\_SCALARS\_FLUX\_BEGIN,MASS\_SCALARS\_FLUX\_END} &
 The indices in the \code{fluxes} data structure used for mass scalars data \htr
 \htabopt{FLUXES\_BEGIN} &
 The first index for the \code{fluxes} data structure \htr
\end{longtable}


\section{Fluid Variables Example}
The snippet of code below shows a \code{Config} file and
parts of a corresponding \code{Simulation.h} file.  

\begin{codeseg}
# Config file for explicit split PPM hydrodynamics.
# source/physics/Hydro/HydroMain/split/PPM  

REQUIRES physics/Hydro/HydroMain/utilities
REQUIRES physics/Eos

DEFAULT PPMKernel

VARIABLE dens TYPE: PER_VOLUME       # density
VARIABLE velx TYPE: PER_MASS         # x-velocity
VARIABLE vely TYPE: PER_MASS         # y-velocity
VARIABLE velz TYPE: PER_MASS         # z-velocity
VARIABLE pres TYPE: GENERIC          # pressure
VARIABLE ener TYPE: PER_MASS         # specific total energy (T+U)
VARIABLE temp TYPE: GENERIC          # temperature
VARIABLE eint TYPE: PER_MASS         # specific internal energy

FLUX rho
FLUX u
FLUX p
FLUX ut
FLUX utt
FLUX e
FLUX eint

SCRATCHVAR otmp
SCRATCHCENTERVAR ftmp
.....
\end{codeseg}

The \code{Simulation.h} files would declare the property variables, fluxes
and scratch variables as:
(The setup script alphabetizes the names.)

\begin{codeseg}
#define DENS_VAR 1
#define EINT_VAR 2
#define ENER_VAR 3
#define PRES_VAR 4
#define TEMP_VAR 5
#define VELX_VAR 6
#define VELY_VAR 7
#define VELZ_VAR 8

#define E_FLUX 1 
#define EINT_FLUX 2 
#define P_FLUX 3
#define RHO_FLUX 4 
#define U_FLUX 5
#define UT_FLUX 6 
#define UTT_FLUX 7 

#define OTMP_SCRATCH_GRID_VAR 1

#define FTMP_SCRATCH_CENTER_VAR 1

\end{codeseg}


%%\subsection{Particle Properties}
\section{Particles }
\label{Sec:FlashHparticles}
\subsection{Particles Types}
\label{Sec:FlashHparttypes}
\flashx now supports the co-existence of multiple particle types
in the same simulation. To facilitate this ability, the particles are
now defined in the \code{Config} files with \code{PARTICLETYPE}
keyword, which is also accompanied by an associated initialization and
mapping method. The following example shows a Config file with passive
particles, and the corresponding generated \code{Simulation.h} lines
\begin{codeseg}

Config file :

PARTICLETYPE passive INITMETHOD lattice MAPMETHOD quadratic ADVMETHOD rungekutta

REQUIRES Particles/ParticlesMain
REQUESTS Particles/ParticlesMain/passive/RungeKutta
REQUESTS Particles/ParticlesMapping/Quadratic
REQUESTS Particles/ParticlesInitialization/Lattice
REQUESTS IO/IOMain/
REQUESTS IO/IOParticles

Simulation.h :

#define PASSIVE_PART_TYPE 1
#define PART_TYPES_BEGIN CONSTANT_ONE
#define NPART_TYPES 1
#define PART_TYPES_END (PART_TYPES_BEGIN + NPART_TYPES - CONSTANT_ONE)

\end{codeseg} 

One line desribing the type, initialization, and mapping methods must be
provided for {\em each} type of particle included in the simulation.

\subsection {Particles Properties}
\label{Sec:FlashHpartprops}
Particle properties are defined within the \code{particles} data structure.
The individual properties will be listed in \code{Simulation.h} if the
\unit{Particles} unit is defined in a simulation.  The variables 
\code{NPART_PROPS}, \code{PART_PROPS_BEGIN} and \code{PART_PROPS_END}
indicate the number and location of particle properties indices.
For example if a Config file has the following specifications
\begin{codeseg}

PARTICLEPROP dens
PARTICLEPROP pres
PARTICLEPROP velx

\end{codeseg}

then the relevant portion of \code{Simulation.h} will contain
\begin{codeseg}
#define DENS_PART_PROP 1
#define PRES_PART_PROP 2
#define VELX_PART_PROP 3
...
#define PART_PROPS_BEGIN CONSTANT_ONE
#define NPART_PROPS 3
#define PART_PROPS_END (PART_PROPS_BEGIN + NPART_PROPS - CONSTANT_ONE)
\end{codeseg}

\section{Non-Replicated Variable Arrays}
\label{Sec:FlashHnonrep}
For each non-replicated variable array defined in the \code{Config} file
(see \secref{Sec:ConfigFileSyntax}), various macros and constants are defined
to assist the user in retrieving the information from the \code{NONREP} line as
well as determining the distribution of array element variables across the meshes.

\subsection{Per-Array Macros}
For each non-replicated variable array \metavar{FOO} as defined by the following
\code{Config} line:

\code{
NONREP \metavar{unktype} FOO \metavar{localmax} \metavar{globalparam} \metavar{ioformat}
}

the following will be placed into \code{Simulation.h}:
\begin{itemize}
\item \code{\#define \metavar{FOO}_NONREP}\\
An integer constant greater than zero to be used as an id for this array.
The user should rarely ever need to use this value.  Instead, it is much more
likely that the user would just query the preprocessor for the existence of
this symbol (via \code{\#ifdef \metavar{FOO}_NONREP}) to enable sections of code.
\item \code{\#define \metavar{FOO}_NONREP_LOC2UNK(loc)}\\
Given the 1-based index into this mesh's local subset of the global
array, returns the \code{UNK} index where that variable is stored.
\item \code{\#define \metavar{FOO}_NONREP_MAXLOCS}\\
The integer constant \metavar{localmax} as supplied in the \code{Config} file
\code{NONREP} line.
\item \code{\#define \metavar{FOO}_NONREP_RPCOUNT}\\
The string literal value of \metavar{globalparam} from the \code{NONREP} line.
\end{itemize}

\subsection{Array Partitioning Macros}
These are the macros used to calculate the subset of indices into the global
variable array each mesh is responsible for:
\begin{itemize}
\item \code{\#define NONREP_NLOCS(mesh,meshes,globs)}\\
Returns the number of elemenets in the mesh's local subset.  This value will always
be less than or equal to \metavar{FOO}_NONREP_MAXLOCS.
\item \code{\#define NONREP_LOC2GLOB(loc,mesh,meshes)}\\
Maps an index into a mesh's local subset to its index in the global array.
\item \code{\#define NONREP_GLOB2LOC(glob,mesh,meshes)}\\
Maps an index of the global array to the index into a mesh's local subset.  If the
mesh provided does not have that global element the result is undefined.
\item \code{\#define NONREP_MESHOFGLOB(glob,meshes)}\\
Returns the mesh that owns a given global array index.
\end{itemize}

Descriptions of arguments to the above macros:
\begin{itemize}
\item \code{loc}:  1-based index into the mesh-local subset of the global array's indices.\\
\item \code{glob}:  1-based index into the global array.\\
\item \code{globs}:  Length of the global array.  This should be the value read from
the array's runtime parameter \metavar{FOO}_NONREP_RPCOUNT.
\item \code{mesh}:  0-based index of the mesh in question.  For a processor, this is
its rank in the \code{MESH_ACROSS_COMM} communicator.\\
\item \code{meshes}:  The number of meshes.  This should be the value of the runtime parameter
\code{meshCopyCount}, or equivalently the number of processors in the
\code{MESH_ACROSS_COMM} communicator.\\
\end{itemize}

\subsection{Example}
In this example the global array \code{FOO} has eight elements, but there are three meshes, so
two of the meshes will receive three elements of the array and one will get two.  How
that distribution is decided is hidden from the user.  The output datasets will contain
variables \code{foo1 \dots foo8}.

Config:
\begin{fcodeseg}
NONREP MASS_SCALAR FOO 3 numFoo foo?
\end{fcodeseg}

flash.par:
\begin{fcodeseg}
meshCopyCount = 3
numFoo = 8
\end{fcodeseg}

Fortran 90:
\begin{fcodeseg}
\#include "Simulation.h"\\
! this shows how to iterate over the UNKs corresponding to this 
! processor's subset of FOO
integer :: nfooglob, nfooloc ! number of foo vars, global and local
integer :: mesh, nmesh ! this mesh number, and total number of meshes
integer :: fooloc, fooglob ! local and global foo indices
integer :: unk\\
call Driver_getMype(MESH_ACROSS_COMM, mesh)
call Driver_getNumProcs(MESH_ACROSS_COMM, nmesh)
call RuntimeParameters_get(FOO_NONREP_RPCOUNT, nfooglob)
nfooloc = NONREP_NLOCS(mesh, nmesh, nfooglob)\\
! iterate over the local subset for this mesh
do fooloc=1, nfooloc
    ! get the location in UNK
    unk = FOO_NONREP_LOC2UNK(fooloc)
    ! get the global index
    fooglob = NONREP_LOC2GLOB(fooloc, mesh, nmesh)\\
    ! you have what you need, do your worst...
end do
\end{fcodeseg}

%%\subsection{Other Preprocessor Symbols}
\section{Other Preprocessor Symbols}
\label{Sec:FlashHother}
The constants \code{FIXEDBLOCKSIZE} and \code{NDIM} are both included
for convenience in this file. \code{NDIM} gives the dimensionality of
the problem, and \code{FIXEDBLOCKSIZE} is defined if and only if fixed blocksize mode is
selected at compile time.

Each \code{Config} file can include the \code{PPDEFINE} keyword to
define additional preprocessor symbols. Each ``\code{PPDEFINE
\metavar{symbol} \metavar{[value]}}'' gets translated to a 
``\code{\#define \metavar{symbol} \metavar{[value]}}''.
This mechanism can be used to write code that depends on which
units are included in the simulation. 
See \chpref{Sec:ConfigFileSyntax} for concrete usage examples.

%\index{FIXEDBLOCKSIZE mode}
