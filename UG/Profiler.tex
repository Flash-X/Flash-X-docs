\chapter{Timer and Profiler Units}
\label{Chp:Profiler and Timer Units}

\section{Timers}\label{Sec:Timers Unit}



\subsection{MPINative}
\label{Sec:MPINative}
Flash-X includes an interface to a set of stopwatch-like timing
routines for monitoring performance.  The interface is defined in the
\code{monitors/Timers} unit, and an implementation that uses the timing 
functionality provided by MPI is provided in \code{monitors/Timers/TimersMain/MPINative}.  
Future implementations might use the PAPI framework to track hardware counter 
details.  


The performance routines start or stop a timer at the beginning or end
of a section of code to be monitored, and accumulate performance
information in dynamically assigned accounting segments.  The code
also has an interface to write the timing summary to the Flash-X
logfile.  These routines are not recommended for timing very short
segments of code due to the overhead in accounting.

There are two ways of using the \code{Timers} routines in your code.
One mode is to simply pass timer names as strings to the start and
stop routines.  In this first way, a timer with the given name will be
created if it doesn't exist, or otherwise reference the one already in
existence.  The second mode of using the timers references them not by
name but by an integer key.  This technique offers potentially faster access if
a timer is to be started and stopped many times (although still not
recommended because of the overhead).  The integer key is obtained by
calling with a string name \api{monitors/Timers/Timers_create} which
will only create the timer if it doesn't exist and will return the
integer key.  This key can then be passed to the start and stop
routines.

The typical usage pattern for the timers is implemented in the default
\code{Driver} implementation.  This pattern is: call \api{monitors/Timers/Timers_init} once 
at the beginning of a run, call \api{monitors/Timers/Timers_start} and
\api{monitors/Timers/Timers_stop} around sections of code, and call
\api{monitors/Timers/Timers_getSummary} at the end of the run to report the timing
summary at the end of the logfile.  However, it is possible to call
\api{monitors/Timers/Timers_reset} in the middle of a run to reset all timing
information.  This could be done along with writing the summary once
per-timestep to report code times on a per-timestep basis, which might
be relevant, for instance, for certain non-fixed operation count
solvers.  Since \api{monitors/Timers/Timers_reset} does not reset the integer key
mappings, it is safe to obtain a key through
\api{monitors/Timers/Timers_create} once in a saved variable, and continue to use it after calling
\api{monitors/Timers/Timers_reset}.


Two runtime parameters control the \code{Timer} unit and are described below.
\begin{center}
\begin{longtable}{lllp{2.7in}}
\caption[parameters]{\label{Tab:timers parameters}Timer Unit runtime parameters.} \\
Parameter                & Type & Default value & Description \\
\hline
\subsequentpageheadings
{\caption[]{Timers parameters (continued).}}
{Parameter                & Type & Default value & Description }
\endhead

\\
\code{eachProcWritesSummary} &  \code{LOGICAL}  & \code{TRUE}            & Should each process write 
	its summary to its own file?  If true, each
        process will write its summary to a file named timer\_summary\_$<$process id$>$
 \\

\code{writeStatSummary} &  \code{LOGICAL}  & \code{TRUE}            & Should timers write the max/min/avg values for timers to the logfile?
        
 \\

\hline
\end{longtable}
\end{center}

\code{monitors/Timers/TimersMain/MPINative} writes two summaries to the logfile: 
the first gives the timer execution of the master
processor, and the second gives the statistics of max, min, and avg
times for timers on all processors.  The secondary max, min, and avg times will not be 
written if some process executed timers differently than another.  For example,
this anomaly happens if not all processors contain at least one block.  In this case, 
the \unit{Hydro} timers only execute on the processors that possess blocks.
See \chpref{Sec:LogfilePerformance} for an example of this type of output.
The max, min, and avg summary can be disabled by setting the runtime parameter
\rpi{Timers/writeStatSummary} to false.  In addition, each process can write its 
summary to its own file named \code{timer\_summary\_$<$process id$>$}.  To prohibit
each process from writing its summary to its own file, set the runtime
parameter \rpi{Timers/eachProcWritesSummary} to false.

\subsection{Tau}
In \code{Flash-X3.1} we add an alternative \code{Timers} implementation
which is designed to be used with the \code{Tau} framework
(\url{http://acts.nersc.gov/tau/}).  Here, we use \code{Tau} API calls
to time the \code{Flash-X} labeled code sections (marked by
\code{Timers\_start} and \code{Timers\_stop}).  After running the
simulation, the \code{Tau} profile contains timing information for
both \code{Flash-X} labeled code sections and all individual subroutines
/ functions.  This is useful because fine grained subroutine /
function level data can be overwhelming in a huge code like
\code{Flash-X}.  Also, the callpaths are preserved, meaning we can see
how long is spent in individual subroutines / functions when they are
called from within a particular \code{Flash-X} labeled code section.
Another reason to use the \code{Tau} version is that the
\code{MPINative} version (See \secref{Sec:MPINative}) is implemented
using recursion, and so incurs significant overhead for fine grain
measurements.

To use this implementation we must compile the \code{Flash-X} source
code with the \code{Tau} compiler wrapper scripts.  These are set as
the default compilers automatically whenever we specify the
\code{-tau} option (see \secref{Sec:ListSetupArgs}) to the setup
script.  In addition to the \code{-tau} option we must specify
\code{--with-unit=\-monitors/\-Timers/\-Timers\-Main/\-Tau} as this \code{Timers}
implementation is not the default.


%CD: Note Quick Reference section speaks about ``modules''.  If we
%uncomment this section, replace all instances of module with
%something like, ``named code section''.
\begin{comment}
\section{Quick Reference}
\label{Sec:timers quick reference}

\subsection {API}
\label {Sec:timers-api}
\begin{itemize}
\item\api{monitors/Timers/Timers\_init}
\item\api{monitors/Timers/Timers\_create}
\item\api{monitors/Timers/Timers\_getSummary}
\item\api{monitors/Timers/Timers\_reset}
\item\api{monitors/Timers/Timers\_start}
\item\api{monitors/Timers/Timers\_stop}
\end{itemize}

All of these routines require a \code{\#include ``Timers\_interface.h''} 
to function properly.  
\end{comment}
\begin{comment}

The list below contains the performance routines along with a short
description of each. Many of the subroutines are overloaded
to take either a name or an integer index.


\begin{itemize}
\item  \code{Timers\_init ()}

Initializes the performance accounting database.
Calls system time routines to subtract out their
initialization overhead.

\item \code{Timers\_create (name,id)}

Creates a timer and returns a unique integer index for the timer.

\item \code{Timers\_start(name)}

Subroutine that begins monitoring code module \code{name} or module
associated with index \code{id}.  If \code{module} is not
associated with a previously assigned accounting
segment, the routine creates one, whereas if \code{id} is not
associated with one, then nothing is done.  The parameter
\code{module} is specified with a string (max 30 characters).
Calling \code{timer\_start} on the same module more than once
without first calling \code{timer\_stop} causes the current
timer for that module to be reset (the accumulated
time in the corresponding accounting segment is not
reset).  Timing modules may be nested as many times
as there are slots for accounting segments (see
MaxModules setting).
The routine may be called with an integer index instead of with the
name of the module.

\item \code{Timer\_stop(name)}

Stops accumulating time in the accounting segment
associated with code module \code{name}. If \code{timer\_stop} is called
for a module that does not exist or for a module that is
not currently being timed, nothing happens.
The routine may be called with an integer index instead of with the
name of the module.



\item \code{perf\_summary (n\_period)}

Subroutine that writes a performance summary of all current
accounting segments to the log file.  Included is the average over
\texttt{n\_period} intervals ({\it e.g.} timesteps).  The accounting
database is not reinitialized.  \code{n\_period} are of default
integer type. Calling \code{perf\_summary} stops all currently
running timers.
\end{itemize}

Below is a very simple example of calling the performance routines.

\begin{codeseg}
       program example
       use perfmon
       integer i
       call timer_init
       do i = 1, 1000
         call timer_start ('blorg')
         call blorg
         call timer_stop ('blorg')
         call timer_start ('gloob')
         call gloob
         call timer_stop ('gloob')
       enddo
       call perf_summary (1000)
       end
\end{codeseg}
\end{comment}

\section{Profiler}\label{Sec:Profiler Unit} 


In addition to an interface for simple timers, Flash-X includes a
generic interface for third-party profiling or tracing libraries.
This interface is defined in the \code{monitors/Profiler} unit.

In \flashx we created an interface to the IBM profiling libraries
libmpihpm.a and libmpihpm\_smp.a and also to HPCToolkit
\url{http://hpctoolkit.org/} (Rice University).  We make use of this
interface to profile Flash-X evolution only, i.e. not initialization.
To use this style of profiling add
\code{-unit=monitors/\-Profiler/\-ProfilerMain/\-mpihpm} or
\code{-unit=monitors/\-Profiler/\-ProfilerMain/\-hpctoolkit} to your
setup line and also set the Flash-X runtime parameter
\code{profileEvolutionOnly} = .true.

For the IBM profiling library (mpihpm) you need to add LIB\_MPIHPM and
LIB\_MPIHPM\_SMP macros to your Makefile.h to link Flash-X to the
profiling libraries.  The actual macro used in the link line depends
on whether you setup Flash-X with multithreading support (LIB\_MPIHPM
for MPI-only Flash-X and LIB\_MPIHPM\_SMP for multithreaded Flash-X).
Example values from sites/miralac1/Makefile.h follow

\begin{codeseg}
LIB_MPI =
HPM_COUNTERS = /bgsys/drivers/ppcfloor/bgpm/lib/libbgpm.a
LIB_MPIHPM = -L/soft/perftools/hpctw -lmpihpm $(HPM_COUNTERS) $(LIB_MPI)
LIB_MPIHPM_SMP = -L/soft/perftools/hpctw -lmpihpm_smp $(HPM_COUNTERS) $(LIB_MPI)
\end{codeseg}

For HPCToolkit you need to set the environmental variable
HPCRUN\_DELAY\_SAMPLING=1 at job launch to enable selective profiling
(see the HPCToolkit user guide).


\begin{comment}
\section{Quick Reference}
\label{Sec:profiler quick reference}

\subsection {API}
\label {Sec:profiler-api}
\begin{itemize}
\item\api{monitors/Profiler/Profiler\_init}
\item\api{monitors/Profiler/Profiler\_getSummary}
\item\api{monitors/Profiler/Profiler\_start}
\item\api{monitors/Profiler/Profiler\_stop}
\end{itemize}
\end{comment}
