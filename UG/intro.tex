
\chapter{Introduction}
\label{Sec:Introduction}

The Flash-X code is a component-based software system for simulation
of multiphysics applications formulated largely as a collection of
partial- and ordinary- differential equations as well as algebraic
equations. The maintained code components are written in a combination of high
level languages such as Fortran, C and C++ with an embedded
domain-specific macro language implemented in the form of key-value
dictionaries. The accompanying configuration tool-chain, written in Python and .... 
can translate and assemble different permutations and combinations of
the components to configure a diverse set of applications.  
% Python should always be capitalized.  See http://docs.python.org/doc/style-guide.html
Also included in the distribution is a accompanying domain-specific
runtime system that can orchestrate data movement between devices
(CPU, accelerators, and other specialized devices that might exist) on a
compute node of a high performance computing (HPC) platform. 
It uses the Message-Passing Interface (MPI) library communication
between nodes, though more than one MPI rank can also be placed on a
node. HDF5 is the default mode for IO. \flashx has three
interchangeable discretization grids: a Uniform Grid, a 
oct-tree based adaptive grid using the \Paramesh
library, and a block-structured adaptive grid using AMReX library,
which also mimics an oct-tree-like layout for use in \flashx.


 