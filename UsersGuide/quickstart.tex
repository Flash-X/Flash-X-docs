
\chapter{Quick Start}
\label{Chp:Quickstart}

This chapter describes how to get up-and-running quickly with Flash-X
with an example simulation, the Sedov explosion.  We explain
how to configure a problem, build it, run it, and examine the output
using IDL.

\section{System requirements}

You should verify that you have the following:

\begin{itemize}

\item A copy of the Flash-X source code distribution. To obtain a copy
  of the source code, please visit https://github.com/Flash-X and
  request access to the Flash-X repository as a collaborator. 

\item A \Fortran compiler, a C compiler, and a C++ compiler, and
  Python 3 or above. 

\item An installed copy of the Message-Passing Interface (MPI)
  library, and the HDF5 library.

  \item To use \amrex for AMR, it must be installed. \amrex requires a
    different version each for 1, 2 or 3 dimensional application
    configurations. 
    (for details see .....). Paramesh is included in the source code.

\item \amrex is compatible with PETSc and Hypre math libraries, and
  \paramesh is compatible with Hypre. They must be installed on the
  platform and their location should be made known to \amrex. 

\item The GNU make utility

\item A copy of the Python language, version 3.0 or later is
  required.
  
\end{itemize}


\section{Obtaining and configuring Flash-X for quick start}
\label{unpack}

To begin, clone the Flash-X source code distribution.....

This quick start example uses \paramesh for AMR and does not depend on
PETSc or Hypre. It also does not use accelerators or other devices on
the node, neither does it exercise advanced features of the
Orchestration System. It requires MPI  and HDF5 installed, and runs
using one CPU per MPI rank.

Once you have the source code and all the necessary libraries installed
configure the Flash-X source tree for the Sedov explosion problem
using the
\code{setup} command
Type \begin{quote} \code{./setup
Sedov -auto} \end{quote} This configures Flash-X for the 2d
\code{Sedov}%\index{setups!sedov}
problem using the default
hydrodynamic solver, equation of state, Grid unit, and I/O format
defined for this problem, linking all necessary files into a new
directory, called `\code{object/}'.  For the purpose of this example, we will
use the default I/O format, serial HDF5.  In order to compile a
problem on a given machine Flash-X allows the user to create a file
called
\code{Makefile.h}%\index{gmake@\code{gmake}!Makefile.h@\code{Makefile.h}}
which sets the paths to compilers and libraries specific to a given
platform.  This file is located in the directory
\code{sites/\metavar{mymachine.my\-inst\-i\-tut\-ion.mydomain}/}.
The \code{setup} script will
attempt to see if your machine/platform has a
\code{Makefile.h}%\index{gmake@\code{gmake}!Makefile.h@\code{Makefile.h}} 
already created, and if so, this will be linked into the
\code{object/} directory.  If one is not created the setup script will use a 
prototype Makefile.h with guesses as to the locations of libraries on your
machine. The current distribution includes prototypes for \emph{Linux}, \emph{OSX} operating systems. In any
case, it is advisable to create a \code{Makefile.h} specific to your
machine. See  
\secref{Sec:SetupMakefile} for details.


Type the command \code{cd object} to enter the object directory which
was created  when you setup the Sedov problem, and then execute
\code{make}. This will compile the Flash-X code. 
\begin{verbatim}
cd object
make
\end{verbatim}
If you have problems and need to recompile, `\code{make
clean}'%\index{make@\code{make}!gmake clean@\code{make clean}}
will remove all object files from the \code{object/} directory,
leaving the source configuration intact; `\code{make
realclean}'%\index{make@\texttt{gmake}!make realclean@\code{makerealclean}}
will remove all files and links from \code{object/}.
After `\code{make realclean}', a new invocation of \code{setup} is
required before the code can be built. The building can take a long
time on some machines; doing a parallel build (\code{make -j} for
example)%\index{make@\code{make}!parallel build}
can significantly
increase compilation speed, even on single processor systems.

Assuming compilation and linking were successful, you should now
find an executable named \code{flashx} in the \code{object/}
directory. You may wish to check that this is the case.

%\vfill
%\eject

If compilation and linking were not successful, here are a few common
suggestions to diagnose the problem:
\begin{itemize}

\item Make sure the correct compilers are in your path, and that they
produce a valid executable.

\item The default Sedov problem uses HDF5 in serial.  Make sure you
  have HDF5 installed. If you do not have HDF5, you can still setup
  and compile Flash-X, but you will not be able to generate either a
  checkpoint or a plot file.  You can setup Flash-X without I/O by
  typing
  \begin{quote}
    \code{./setup Sedov -auto +noio}
  \end{quote}

\item Make sure the paths to the MPI and HDF libraries are correctly set
  in the \code{Makefile.h} in the \code{object/} directory.

\item Make sure your version of MPI creates a valid executable that
  can run in parallel.

\end{itemize}


Flash-X by default expects to find a text file named
\code{flash.par}%\index{flash.par@\code{flash.par}}
in the directory
from which it is run. This file sets the values of various runtime
parameters that determine the behavior of Flash-X. If it is not
present, Flash-X will abort; \code{flash.par} must be created in order
for the program to run (note: all of the distributed setups already
come with a \code{flash.par} which is \emph{copied} into the
\code{object/} directory at setup time).  There is command-line option to use a
different name for this file, described in the next section. Here we
will create a simple \code{flash.par} that sets a few parameters and
allows the rest to take on default values. With your text editor,
edit the \code{flash.par} in the \code{object} directory so it looks
like \figref{Fig:flash.par}.

\begin{figure}[htbp]
\begin{shrink}
\begin{fcodeseg}
# runtime parameters

basenm  = "sedov_"

lrefine_max = 5
refine_var_1 = "dens"
refine_var_2 = "pres"

restart = .false.
checkpointFileIntervalTime = 0.01

nend = 10000
tmax = 0.05

gamma = 1.4

xl_boundary_type = "outflow"
xr_boundary_type = "outflow"

yl_boundary_type = "outflow"
yr_boundary_type = "outflow"

plot_var_1 = "dens"
plot_var_2 = "temp"
plot_var_3 = "pres"

sim_profFileName = "/dev/null"
\end{fcodeseg}
\end{shrink}
\caption{\label{Fig:flash.par} Flash-X parameter file contents for the
quick start example.}
\end{figure}

This example first instructs Flash-X to 
name output files appropriately (through the \code{basenm} parameter).
Flash-X is then instructed to use up to five levels of adaptive
mesh refinement (AMR) (through the \code{lrefine\_max} parameter),
with the actual refinement adapted based on the density and pressure of the solution
(parameters \code{refine\_var\_1} and \code{refine\_var\_2}). We will
not be starting from a checkpoint file 
(``\code{restart = .false.}'' ---
this is the default, but here it is explicitly set for clarity).
Output files are to be written every 0.01 time units
(\code{checkpointFileIntervalTime}) and will be created until
$t=0.05$ or 10000 timesteps have been taken (\code{tmax} and
\code{nend} respectively), whichever comes first. The ratio of
specific heats for the gas (\code{gamma} = $\gamma$) is taken to be 1.4, and all
four boundaries of the two-dimensional grid have outflow
(zero-gradient or Neumann) boundary conditions (set via the
\code{\metavar{$[\code{x}\code{y}][\code{l}\code{r}]$}\_boundary\_type} parameters).

Note the format of the file -- each line is of the form
\metavar{variable} = \metavar{value}, a comment (denoted by a
hash mark, \code{\#}), or a blank.  String values are enclosed in double quotes
(\code{"}). Boolean values are indicated in the \FORTRAN\ style,
\code{.true.} or \code{.false.}. Be sure to insert a carriage return
after the last line of text.  A full list of the parameters available
for your current setup is contained in the file \code{setup\_params} 
located in the \code{object/} directory, which
also includes brief comments for each parameter.  If you wish to skip
the creation of a \code{flash.par}, a complete example is provided in
the \code{source/Simulation/SimulationMain/Sedov/} directory.

\section{Running Flash-X}

We are now ready to run Flash-X. To run Flash-X on \emph{N} processors,
type 
\begin{quote} 
\code{mpirun -np }\emph{N\/}~\code{flash}\emph{X}
\end{quote}

\noindent remembering to replace \emph{N} and \emph{X} with the
appropriate values. Some systems may require you to start MPI programs
with a different command; use whichever command is appropriate for
your system. The \flashx executable accepts an optional command-line
argument for the runtime parameters file. If 
``\code{-par\_file}~\metavar{filename}''
is present, Flash-X reads the file specified on command 
line for runtime parameters, otherwise it reads \code{flash.par}.  

%%file (\code{-par\_file} \latexhtml{$<$}{&lt;}filename\latexhtml{$>$}{&gt;}).

\begin{comment}
, and the name of
checkpoint file (\code{-chk\_file}
\latexhtml{$<$}{&lt;}filename\latexhtml{$>$}{&gt;}).
\end{comment}

\begin{comment}
If the \code{-chk\_file} is found, Flash-X assumes that it is starting from
restart, and it gets the runtime parameters from the file specified on
the command line. In this situation \emph{flash.par} is ignored. If
both the arguments are present, the \code{\-par\_file} argument is
ignored and runtime parameters are read from the checkpoint file
specified on command line.
\end{comment}

You should see a number of lines of output indicating that Flash-X is 
initializing the Sedov problem, listing the initial parameters, and 
giving the timestep chosen at each step. After the run is finished, 
you should find several files in the current directory:
\begin{itemize} 
\item \code{sedov.log}%\index{sedov.log@\code{sedov.log}}
  echoes the
    runtime parameter settings and indicates the run time, the build
    time, and the build machine. During the run, a line is written
    for each timestep, along with any warning messages. If the run
    terminates normally, a performance summary is written to this
    file. 
\begin {comment}
Messages indicating when the code refined and what output
    resulted are also contained in \code{sedov.log}.  
\end{comment}
\item \code{sedov.dat}%\index{sedov.dat@\code{sedov.dat}}
  contains a
      number of integral quantities as functions of time: total
      mass, total energy, total momentum, \etc  This file can
      be used directly by plotting programs such as \code{gnuplot};
      note that the first line begins with a hash (\code{\#}) and is
      thus ignored by \code{gnuplot}.  

\item \code{sedov\_hdf5\_chk\_000*} are the different checkpoint
       files. These are complete dumps of the entire simulation state at
        intervals of \code{checkpointFileIntervalTime} 
	and are suitable for use in
        restarting the simulation. \item
\code{sedov\_hdf5\_plt\_cnt\_000*} are plot files. In this example, these 
        files contain density, temperature, and pressure in single
precision.  If needed, more variables can be dumped in the plotfiles
by specifying them in \emph{flash.par}. They are usually
written more frequently 
than checkpoint files, since they are 
the primary output of Flash-X for analyzing the results of the
simulation. They are also used for making simulation
movies. Checkpoint files can also be used for analysis and sometimes
it is necessary to use them since they have comprehensive information
about the state of the simulation at a given time. However, in general,
plotfiles are preferred since they have more frequent snapshots of the
time evolution. Please see \chpref{Chp:IO} for more information about IO outputs.
\end{itemize}


Flash-X is intended to be customized by the user to work with
interesting initial and boundary conditions.  In the following
sections, we will cover in more detail the algorithms and structure of
Flash-X and the sample problems and tools distributed with it.
