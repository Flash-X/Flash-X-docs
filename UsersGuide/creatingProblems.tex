\chapter{Setting Up New Problems}
\label{Chp:Creating new problems}

A new Flash-X problem\index{Simulation\_initBlock@\code{Simulation\_initBlock}!creating new}
is created by making a directory for it under
\code{Flash-X4/\-source/\-Simulation/\-SimulationMain}. This location is where
the Flash-X \code{setup} script looks to find the problem-specific
files.  The Flash-X distribution includes a number of pre-written
simulations; however, most Flash-X users will want to simulate their own
problems, so it is important to understand the techniques for adding a
customized problem simulation.

Every simulation directory contains routines to initialize the
Flash-X grid.  The directory also includes a \code{Config} file which
contains information about infrastructure and physics units, and the
runtime parameters required by the simulation
(see \chpref{Chp:The Flash-X configuration script}).
The files that are usually included in the Simulation directory for
a problem\index{Simulation\_initBlock@\code{Simulation\_initBlock}}
\index{Config@\code{Config}} are:

% make extra big spaces between filesI thought
%\renewcommand{\arraystretch}{1.5}
\begin{tabular}{lp{4.0in}}
\code{Config} &     Lists the units and variables required for the problem, defines
                    runtime parameters and initializes them with
default values.  \\
\code{Makefile} & The \code{make} include file for the Simulation. \\
\code{Simulation\_data.F90} & Fortran module which stores data and parameters specific to the Simulation.  \\
\code{Simulation\_init.F90} & Fortran routine which reads the runtime
parameters, and performs other necessary initializations.  \\
\code{Simulation\_initBlock.F90} & Fortran routine for setting initial conditions in a
single block.  \\
\code{Simulation\_initSpecies.F90} & Optional Fortran routine for
initializing species properties if multiple species are being used. \\
\code{flash.par} & A text file that specifies values for the runtime
parameters. The values in flash.par override the defaults from Config
files.  \\
\end{tabular}
\renewcommand{\arraystretch}{1.0}

In addition to these basic files, a particular simulation may include
some files of its own. These files could provide either new
functionality not available in Flash-X, or they may include customized
versions of any of the Flash-X routines. For example, a problem might
require a custom refinement criterion instead of the one provided
with Flash-X. If a customized implementation of
