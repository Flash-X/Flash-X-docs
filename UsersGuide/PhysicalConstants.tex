\chapter{Physical Constants Unit}
\label{Chp:PhysicalConstants} %\index{Physical Constants}



The Physical Constants unit provides a set of common constants, such
as Pi and the gravitational constant, in various systems of 
measurement units.
The default system of units is CGS, so named for having a length
unit in centimeters, a mass unit in grams, and a time unit in
seconds. In CGS, the charge unit is the esu, and the temperature
unit is the Kelvin. %\index{Physical Constants!CGS,CGS}
The constants
can also be obtained in the standard MKS system of units, where
length is in meters, mass in kilograms, and time in seconds. For MKS
units, charge is in Coloumbs, and temperature in Kelvin.
%\index{Physical Constants!MKS,MKS}  

\begin{flashtip}
For ease of usage, the constant PI=3.14159.... is defined in 
the header file \code{constants.h}.  Including this file with
\#include ``constants.h'' is an
alternate way to access the value of $\pi$, rather than needing
to include the \code{PhysicalConstants} unit.
\end{flashtip}

Any constant can optionally be converted from the standard units
into any other available units. This facility makes it easy to
ensure that all parts of the code are using a consistent set of
physical constant values and unit conversions.

For example, a program using this unit might obtain the
value of Newton's gravitational constant $G$ in units of
Mpc$^3$~Gyr$^{-2}M_\odot^{-1}$ by calling
\begin{quote}
\begin{codeseg}
call PhysicalConstants_get ("Newton", G, len_unit="Mpc",
                       time_unit="Gyr", mass_unit="Msun")
\end{codeseg}
\end{quote}
In this example, the local variable \code{G} is set equal to the result,
$4.4983\times 10^{-15}$ (to five significant figures).

Physical constants are taken from K. Nahamura {\it et al.} (Particle
Data Group), J. Phys. G {\bf 37}, 075021 (2010). 

%%  Newer data are available! See
%%Peter J. Mohr and Barry N. Taylor (January 2005). "CODATA recommended values of the
%%fundamental physical constants: 2002". Reviews of Modern Physics 77: 1-107. An in-depth 
%%discussion of how the CODATA constants were selected and determined.
%% More generally, <http://pdg.lbl.gov>

\section{Available Constants and Units}
\label{sec:PhysicalConstantsAvailable}

There are many constants and units available within \flashx,
see
\tblref{Tab:PhysicalConstants} and
\tblref{Tab:PhysicalConstantsUnits}. Should the user wish to add
additional constants or units to a particular setup, the routine
\api{PhysicalConstants/PhysicalConstants_init} should be
overridden and the new constants added within the directory of the setup.

\begin{table}
\caption{Available Physical Constants}
\label{Tab:PhysicalConstants}
\begin{center}
\begin{tabular}{|ll|} \hline
{\bf String Constant} & {\bf Description} \\ \hline 
        Newton &      Gravitational constant G \\
        speed of light &     Speed of light \\
        Planck &        Planck's constant \\
        electron charge &      charge of an electron \\
        electron mass &        mass of an electron \\
        proton mass &          Mass of a proton \\
        fine-structure &      fine-structure constant \\
        Avogadro &            Avogadro's Mole Fraction \\
        Boltzmann &            Boltzmann's constant \\
        ideal gas constant &   ideal gas constant \\
        Wien &              Wien displacement law constant \\
        Stefan-Boltzmann &  Stefan-Boltzman constant \\
        pi &          Pi \\
        e &           e \\
        Euler &  Euler-Mascheroni constant \\ \hline
\end{tabular}

\end{center}

\end{table}

\begin{table}[ht]  % really, we want it here!
\caption{Available Units for Conversion of Physical Constants}
\label{Tab:PhysicalConstantsUnits}
\begin{center}
\begin{tabular}{|llll|}  \hline
{\bf Base unit} & \textbf{String Constant} & \textbf{Value in CGS units} & \textbf{Description} \\ \hline
length &         cm & 1.0 & centimeter \\
time &            s &     1.0 & second \\
temperature &     K &     1.0 & degree Kelvin \\
mass &            g &     1.0 & gram \\
charge &          esu &   1.0 & ESU charge \\
length &          m &      \code{1.0E2}  & meter \\
 length &          km &    \code{1.0E5}    & kilometer \\
 length &          pc &    \code{3.0856775807E18} & parsec \\
 length &          kpc &   \code{3.0856775807E21} & kiloparsec \\
 length &          Mpc &   \code{3.0856775807E24} & megaparsec \\
 length &          Gpc &   \code{3.0856775807E27} & gigaparsec \\
 length &          Rsun &  \code{6.96E10} & solar radius \\
 length &          AU &    \code{1.49597870662E13} & astronomical unit \\
 time &            yr &    \code{3.15569252E7} & year \\
 time &            Myr &   \code{3.15569252E13} & megayear \\
 time &            Gyr &   \code{3.15569252E16} & gigayear\\
 mass &            kg &    \code{1.0E3}  & kilogram \\
 mass &            Msun &  \code{1.9889225E33} & solar mass \\
 mass &            amu &   \code{1.660538782E-24} & atomic mass unit \\
 charge &          C &     \code{2.99792458E9}  & Coulomb \\ 
        &            &                          &         \\
\multicolumn{4}{|c|}{Cosmology-friendly units  using \(H_0 = 100\) km/s/Mpc:} \\ %\hline
 length &          LFLY &   \code{3.0856775807E24} & 1 Mpc \\
 time &            TFLY &   \code{2.05759E17} & \(\frac{2}{3H_0}\) \\
 mass &            MFLY &   \code{9.8847E45} & \code{5.23e12} Msun \\ \hline
\end{tabular}

\end{center}
\end{table}

\section{Applicable Runtime Parameters}
\label{sec:PhysicalConstantsRP}

There is only one runtime parameter used by the Physical Constants
unit:
\rpi{PhysicalConstants/pc_unitsBase} selects the default system of units for
returned constants.  It is a three-character string set to "CGS" or
"MKS";  the default is CGS.


\section{Routine Descriptions}
\label{sec:PhysicalConstantsRoutines}

The following routines are supplied by this unit.

\begin{itemize}
\item
    \api{PhysicalConstants/PhysicalConstants_get}
    Request a physical constant given by a string, and
    returns its real value.  This routine takes
    optional arguments for converting units from the default.  If
    the constant name or any of the optional unit names aren't recognized, a
    value of 0 is returned.

\item
    \api{PhysicalConstants/PhysicalConstants_init}
    Initializes the Physical Constants Unit by loading all
    constants.  This routine is called by \api{Driver/Driver_initFlash} and must be called before the
    first invocation of \code{Physical\-Constants\_get}.
    In general, the user does not need to invoke this call.

\item
    \api{PhysicalConstants/PhysicalConstants_list}
    Lists the available physical constants in a snappy table.

\item
    \api{PhysicalConstants/PhysicalConstants_listUnits}
    Lists all the units available for optional conversion.

\item
    \api{PhysicalConstants/PhysicalConstants_unitTest}
    Lists all physical constants and units, and tests the unit conversion
    routines.

\end{itemize}


\begin{flashtip}
The header file \code{PhysicalConstants.h} must be included in the
calling routine due to the optional arguments of
\code{PhysicalConstants\_get}.
\end{flashtip}

\section{Unit Test}
\label{Sec:PhysConstUnitTest}
The \code{PhysicalConstants} unit test 
\api{PhysicalConstants/PhysicalConstants_unitTest}
is a simple exercise of the functionality 
in the unit.  It does not require time stepping or the grid.
``Correct'' usage is indicated, as is erroneous usage.

